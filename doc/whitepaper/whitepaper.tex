\documentclass[twoside]{whitepaper}
\usepackage{cite, xspace}
\usepackage{color}
\newcommand{\Pistachio}{L4Ka::Pistachio\xspace}
\newcommand{\Hazelnut}{L4Ka::Hazelnut\xspace}
\newcommand{\Lx}{L4Ka::Linux\xspace}
\newcommand{\IDL}{L4Ka::IDL4\xspace}

\title{{\Huge The \Pistachio Exokernel}\\white paper}
\author{System Architecture Group\\University of Karlsruhe}

\definecolor{headerbg}{gray}{.9}
\newcommand{\headerbox}[1]%
{\hspace*{-.165\hsize}\colorbox{headerbg}%
{\makebox{\hspace*{.1\hsize}\rule{0pt}{.4ex}\hspace*{#1em}}}%
\hspace*{-#1em}}

\newcommand{\headerboX}%
{\hspace*{-.162\hsize}\colorbox{headerbg}%
{\makebox{\hspace*{1.1\hsize}\rule{0pt}{.2ex}}}%
\hspace*{-1\hsize}}

\begin{document}
\sffamily
\maketitle

\section{\Pistachio}

\Pistachio is now an L4 exokernel developed by the System
Architecture Group at the University of Karlsruhe.  Unlike the earlier
microkernel, the exokernel exports only minimal mechanisms and leaves
resource management to user-level servers.  It still implements the
L4 Version 4 kernel API, is fully 32 and 64 bit clean, provides
multiprocessor support, and offers superfast local IPC.  The design
encourages custom resource managers that implement scheduling and
memory policies outside the kernel.

The first release implements most of the core
functionality of the API.  The kernel is written in modern C++\@ (currently
C++23) with a significant focus on performance, portability, and reusability.
\Pistachio supports most existing mainstream hardware architectures,
in particular Intel's IA32\footnote{Pentium and higher}, and
IA64\footnote{Itanium1 and SKI}, PowerPC 32bit, Alpha 21164, and MIPS
R4000 and higher, with multiprocessor support for the first three
architectures.  In the near future additional support is planned for
AMD64, Power4, ARM\footnote{StrongARM and XScale}, and UltraSparc.

\Pistachio provides a solid basis for research and development of
highly specialized and general purpose operating systems for a wide
variety of systems, ranging from tiny embedded devices to huge
multiprocessor server systems.  As an example, \Lx, a modified Linux 2.4 kernel running on top of \Pistachio,
enables unmodified commodity applications to coexist with customized 
system components, which allows a smooth migration path towards customized
services, and supports secure extensions to monolithic operating systems.  
This is a core requirement for upcoming consumer devices such as
SmartPhones, secure payment, and digital rights management.


\section{L4 Version 4 API}
\Pistachio implements the L4 Version 4 API (currently still referred to
as eXperimental Version 2, or X2).  The Version 4 API supersedes the
seven year old Version 2 API (V2), which was the basis for L4's success
story.  All L4 APIs are designed with the main focus on performance
and flexibility.  As described below, Version 4 introduces a set of
new kernel features eliminating limitations experienced with V2.

\subsection{Separation of API and ABI}
L4 kernels have been implemented for multiple architectures,
usually in an ad-hoc way, and often purely in assembly.  Thus
application binary interfaces (ABIs) were defined by the specific
implementation, leading to different behavior and kernel features
loosely following the initial IA32 reference implementation.  
As a result, system software was very specialized for one particular
kernel implementation and therefore inherently non-portable.

Version 4 enforces a strict separation of API and ABI to support
portable, architecture independent system software.  The API specifies
a set of high-level language bindings which are generic across
all architectures.

\subsection{Virtual Registers}
Today's widely used processor architectures have very different
register sets, featuring as few as seven general purpose registers on
IA32 up to 128 integer and 128 floating point registers on IA64.  

L4's IPC performance is achieved by strictly avoiding unnecessary
memory references, thereby reducing cache and TLB footprint.  However,
having a single generic API for such very different architectures as
IA32 and IA64 would have sacrificed performance for one or the other.

Version 4 therefore introduces the notion of a virtual register file per thread.
Depending on processor characteristics, these registers map either to
memory or to actual processor registers, allowing optimal use of hardware
registers on all architectures.

\subsection{Address Space and Threads}
In Version 4 the fixed relation between address spaces and threads is
removed.  Any thread can be a member of any address space and threads
can be transparently migrated between spaces.  On 32 bit architectures
up to $2^{18}$ and on 64 bit architectures up to $2^{32}$ concurrent threads
are supported.

\subsection{Kernel Interface Page}
Rapidly increasing processor clock speeds require architectural
extensions and modifications from one processor generation to the
next.  A recent example is the introduction of the SYSENTER/SYSEXIT instructions for IA32,
which allow up to 10$\times$ faster kernel entry and exit compared to
the original software interrupt method.  However, to utilize this new
feature all legacy code has to be recompiled.

Version 4 introduces a kernel provided page, the kernel interface page
(KIP), which contains function entry points for all system calls as well
as frequently accessed system information.  The invocation of a
system call is now a simple function call to the entry point in the
KIP.  When new architectural features are added to next generation
processors, these features can be directly utilized by replacing the
page; old code will automatically use the latest processor features.

This methodology also allows for system calls which do not necessarily
enter the kernel, e.g., the system clock or super-fast local IPC.  The
overhead of this flexible scheme is negligible when using an optimized
dynamic linker which links system calls against the KIP's entry
points.

\subsection{SuperFast IPC}
L4 provides a very limited yet extremely flexible and powerful
set of abstraction and mechanisms: threads, address spaces, IPC, and
mapping.  All other primitives are built upon these four basic
abstractions, including synchronization primitives.

Even though L4's IPC is extremely fast, it is still limited by the
hardware architecture.  The invocation of an IPC requires a change of
the privilege mode, a costly operation on most hardware architectures.
In many cases, however, IPC is used for synchronization and signaling
of threads executing within the same address space.
IPC can then be performed completely in user mode avoiding the overhead
induced by the two unnecessary privilege level changes.  First experiments
have shown that it is possible to achieve an order of magnitude higher
IPC performance.\footnote{In the current release \Pistachio does not yet implement
SuperFast IPC.}

\subsection{Multi-Processor Support}
The Version 4 API has integral support for multi-processor systems,
with its main focus on scalability.  Load balancing and scheduling
decisions are purely user-level based and thus support the
implementation of arbitrary processor allocation policies.

A fundamental prerequisite for SMP scalability is the preservation
of parallelism of user applications within the kernel, i.e., the kernel
must not serialize operations of two parallel threads.  The Version 4 API
strictly follows this design rule by providing powerful orthogonal
interfaces.

NUMA systems are supported with the aforementioned flexible thread
allocation primitives.  Thread migration allows use of node local
memory and thus increases overall system performance.  Similar
optimizations are possible for simultaneous multi-threaded
architectures, such as Intel's Hyper-Threading technology.

\Pistachio's SMP support is experimental and not fully scalable yet due
to a few coarse-grained locks, in particular in the memory subsystem.
However, the critical IPC path is fully lock-free giving a good first
performance indication.  Multiprocessor support is available for IA32
and IA64 systems.

\subsection{Interrupts}
Similar to the Version 2 API, Version 4 abstracts system interrupts as kernel
threads and interrupt delivery as IPC.  Version 4, however, completely
abstracts the first-level interrupt controller and only provides basic primitives
for interrupt association.  This allows for a higher level of parallelism, more efficient
synchronization, and caching of interrupt controller state thereby
increasing overall performance.  Abstracting interrupt hardware was
also a portability requirement, since some architectures allow access
to interrupt controllers only in privileged mode.

\subsection{Privileged Threads}
Version 4 defines three privileged tasks: sigma0, sigma1, and the root
task.  All threads of a privileged task are allowed to execute certain
system calls prohibited to non-privileged tasks.  This
scheme exports rights delegation completely to user level and
supports implementation of arbitrary policies, such as access control
lists, or capabilities using remote procedure calls (RPC).


\section{Tools}
Powerful development tools can significantly shorten the development
cycle.  Together with \Pistachio we release \IDL, an IDL compiler
which generates RPC stub code that is highly optimized for the kernel API and ABI,
the hardware architecture, and the C/C++ compiler.  \IDL's generated code
quality is comparable to hand-optimized assembly stub code.

\IDL not only supports the Version 4 API but also V2 and X.0 allowing a
smooth migration path for existing software.  It supports the CORBA
and DCE syntax and has an integrated C++ parser which can parse
C/C++ header files to, e.g., re-use type declarations.

The specification of Version 4's IPC interface is significantly
influenced by and optimized for \IDL.  Providing a large memory-based
register file usually has a significant impact on the cache footprint of
message transfers.  However, using specifically optimized marshaling
stubs the additional costs can be completely eliminated resulting in
better overall performance.

RPC takes place as a three-step process.  In the first step all
parameters are marshaled into a message buffer.  This message buffer
is then transfered using the system's communication primitives and finally
un-marshaled at the destination.  The Version 2 API provides two message
registers, X.0 up to three; messages exceeding this limit have to be
transfered using a memory copy with a significant startup overhead.
Version 4 provides up to 64 message registers with some of them
backed by memory.  The IDL compiler can marshal the parameters directly
into these message registers and the kernel can transfer them from
one address space to another with significantly less overhead.  On the
receiver's side the parameters can be directly used from the message
register store.

\end{document}
