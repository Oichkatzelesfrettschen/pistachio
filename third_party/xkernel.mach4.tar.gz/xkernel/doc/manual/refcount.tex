%
% refcount.tex
%
% $Revision: 1.16 $
% $Date: 1993/11/18 06:05:28 $
%

\section{Reference Counting XObjs}
\label{refcnt}

\def\xPush{{\em xPush}}
\def\xPop{{\em xPop}}
\def\xCall{{\em xCall}}
\def\xCallPop{{\em xCallPop}}
\def\xDemux{{\em xDemux}}
\def\xOpen{{\em xOpen}}
\def\xClose{{\em xClose}}
\def\xDuplicate{{\em xDuplicate}}
\def\xCreateSessn{{\em xCreateSessn}}

Version 3.2 of the \xk{} maintains reference counts for sessions in
order to facilitate their destruction. The following implementation
notes should guide the protocol writer towards programming techniques
that are consistent with the reference count mechanism.

In v3.1, the {\em XObj} structure had a reference count field which
was supposed to keep track of the number of external references to
that object.  The intent was that the object could safely be destroyed
when the reference count fell to zero.

Unfortunately, the proper use of reference counts was never
documented.  With each protocol writer deciding to handle session
reference counts in a different manner (if at all), reference counts
were seldom accurate.  This resulted in idle sessions that were never
freed and in dangling references to sessions which had been destroyed.

Constructing a protocol which directly manipulates reference counts
and does so correctly can be awkward and tedious.  Version 3.2 of the
\xk{} has addressed this difficulty by:

\begin{itemize}
\item{}
moving all direct reference count manipulation into the \xk{}
infrastructure.  
\item{}
changing the semantics of some of the UPI functions
\item{}
explicitly documenting how references to other {\em XObj}s may
be used.
\item{}
adding system support for session caching and garbage collection. 
\end{itemize}

The rest of this section discusses the system support for reference
counts and explains how to modify a v3.1 protocol to properly use
reference counts in v3.2.

We discuss {\em XObj} reference counts in the context of
sessions (rather than protocols) because protocols are relatively
static objects and the issues surrounding their reference counts are
not very interesting.


\subsection{References}

An {\em XObj} reference is a pointer to an {\em XObj}.  References come in
two flavors: {\sl permanent} and {\sl temporary}.  

A permanent reference can be used indefinitely.  As long as the holder
of a permanent reference does not call \xClose{} on the reference, the
holder knows that the pointer will remain valid.  Permanent references
are returned by \xOpen{}.

An {\em XObj} pointer received as a parameter in a function call is a
temporary reference.  It can only be safely used for the duration of
that function call.

A permanent reference may be created from a temporary reference by
using the UPI function \xDuplicate{}.  For example, \xDemux{} passes a
reference to the lower session as a parameter to the upper protocol.
If the upper protocol wishes to save this reference beyond the extent
of the demux call, the protocol should call \xDuplicate{} on the
session.  The protocol can then safely use the reference until it
calls \xClose{}.


\subsection{Reference counts}

Session reference counts in v3.2 are a sum of:
\begin{itemize}
\begin{itemize}
\item{}
the number of permanent external references to the session
\item{}
the number of outstanding \xPop{}'s on the session
\end{itemize}
\end{itemize}

\subsubsection{ Counting External References }

Several of the UPI functions are involved in maintaining the count of
permanent external references.  They are listed here along with their
semantics with respect to reference counts (and how these semantics
have changed from \xk{} version 3.1):

\begin{quote}
\begin{em}
\noindent xCreateSessn( ... )
\end{em}
\begin{quote}

\noindent The newly created session has an initial reference
count of 0 (it was 1 in v3.1.)

\end{quote}
\end{quote}


\begin{quote}
\begin{em}
\noindent xOpen( ... )
\end{em}
\begin{quote}

\noindent The session returned by invoking the lower protocol's
open routine has its reference count incremented
before it is returned to the caller of \xOpen{}, indicating that the
caller now has a permanent reference to the session.  (\xOpen{}
did not manipulate reference counts in v3.1.)

\end{quote}
\end{quote}

\begin{quote}
\begin{em}
\noindent xClose( XObj s )
\end{em}
\begin{quote}
\noindent Decrements the reference count of the session, indicating
that a permanent reference to the session has been released.  
The lower protocol's close operation is called {\sl only} if
the session's new reference count is zero.  (\xClose{} always
invoked the lower protocol's close operation in v3.1.) 

\end{quote}
\end{quote}

\begin{quote}
\begin{em}
\noindent xDuplicate( XObj s )
\end{em}

\begin{quote}

\noindent By default, increments the session's reference count,
indicating that a new permanent reference to the session has been
created.  If the session has its own duplicate function, that is
called instead.  (\xDuplicate{} did not exist in v3.1.)

\end{quote}
\end{quote}

\subsubsection{ Counting xPops }

To understand the second component of the reference count, the number
of outstanding \xPop{}s on a session, it is important to realize that
a protocol does not have explicit references to its own sessions.
That is, while a protocol usually maintains pointers to its sessions
(in a map) and invokes operations on them, the sessions' reference
counts do not take these pointers into account.  (This is why
\xCreateSessn{} returns an object with a reference count of 0 and not 1.)

For a protocol to safely send this session pointer outside of the
protocol (e.g., as a parameter in \xDemux{}), it needs to turn the
pointer into a reference by incrementing the session's reference
count.  It could do this by calling \xDuplicate{} on the session as
soon as it is extracted from the session map, making the external
call, and then closing the session.  But since all protocols must do
this for incoming messages, this functionality has been absorbed into
\xPop{}. \xPop{} increments the reference count of the session before 
calling the pop function and decrements it (possibly calling the
session's close operation) afterwards.

It could be argued that all UPI functions, not just \xPop{}, should
indicate their use of a session by incrementing the reference count at
their start and decrementing it at their completion.  To perform any
other UPI operation on a session, however, one must already have a
reference to that session.  As long as an \xClose{} is not performed
on that reference, the session is not going to go away; maintaining
reference counts for these operations is not necessary. \xPop{}s are
different in that a session's reference count does not reflect that
its protocol may send messages up through it.

Note that reference counts will not help a protocol which performs an
\xClose{} on a session reference while another thread has an
outstanding operation on that same reference.  To perform such a
sequence is a protocol error.  If two threads share a session
reference, they should either synchronize to avoid such sequences or
they should duplicate the reference with \xDuplicate{} and each thread
should \xClose{} its reference when it is through.


\subsection{Internal vs. External Reference Counts}

Session reference counts are meant to count the number of {\em
external} references to the session (i.e., references held outside of
the protocol.)  If a protocol must keep track of session references
internal to the protocol itself, a separate mechanism must be used.

This requirement is driven by upper protocols which require that when
they release all of their references to a lower session that the
session's reference count goes to zero.  Virtual protocols which
manipulate lower session's upper protocol pointers are specific
examples of these upper protocols.


\subsection{Session Caching Strategies}

In the presence of correct management of reference counts, protocols
may want to implement some sort of session caching.  For example, many
of the sessions created on the receiving end of a remote procedure
call will have their reference counts go to zero after the reply has
been sent, and without caching, all of those sessions will be
destroyed.  While caching may not be important for all protocols, it
is vital to the performance of protocols that experience a high
frequency of traffic from the same sources.

If a protocol does cache idle sessions, it is important to make the
caching transparent to upper protocols (i.e., an upper protocol should
not be able to distinguish a newly created lower session from a
cached lower session that is being reused.)  One aspect of this
requirement is that protocols which cache sessions must be sure to
check for the presence of openEnables when reusing sessions which
would otherwise be passively created and to 
call xOpenDone when reusing such sessions from the cache.

Here are some examples of caching strategies and how a protocol might
implement them.  These are just examples -- other strategies are
certainly possible:

\begin{itemize}

\item{}
Destroy the session if inactive.  

In this ``null caching'' strategy, the session's close routine
destroys the session:

\begin{verbatim}
        foo_close(s)
                xAssert(s->rcnt == 0);
                xDestroy(s);
\end{verbatim}


\item{}
Garbage collector on a per-session basis

A garbage collector event is started for each idle session.  When
the event expires, the session is collected.  

\let\tt=\COURIERtt
\begin{verbatim}
        foo_demux()
                if ( active session does not exist ) {	
                        if ( openEnable exists ) {
                                if ( cached session exists ) {
                                        evCancel(sstate->gc);
                                        sstate->gc = 0;
                                } else {
                                        s = foo_open();
                                }
                                xOpenDone(s, ...);
                        } else {
                                drop packet; return;
                        }
                }
                xPop()

        foo_open()
                ...
                if ( active session does not exist ) {
                        if (cached session exists ) {		
                                evCancel(sstate->gc);
                                sstate->gc = 0;
                        }
                }
                ...
        
        foo_close(s)
                sstate->gc = evRegister(foo_gc, s);
                mark session as cached

        foo_gc(s)
                xDestroy(s);
\end{verbatim}
\let\tt=\CMRtt

\item{}
Garbage collector on a per-protocol basis

    There is a single garbage collector for the protocol---it runs
    every so often and looks for idle sessions in the protocol's
    session map, calling a protocol-specific destroy function if it
    finds one.  

    The \xk{} has a default session garbage collector which a
    protocol can use for this purpose.  See xkernel/include/gc.h for 
    the interface.

\begin{verbatim}
        foo_init()
                ...
                initSessionCollector(cacheMap, interval, foo_destroy);

        foo_demux()
                if ( active session does not exist ) {	
                        if ( openEnable exists ) {
                                if ( cached session exists ) {
                                        move sessn from cached map to
                                        active map
                                } else {
                                        s = foo_open();
                                }
                                xOpenDone(s, ...);
                        } else {
                                drop packet; return;
                        }
                }
                xPop()
                        
        foo_open()
                ...
                if ( active session does not exist ) {
                        if (cached session exists ) {		
                                move sessn from cached map to active map
                        }
                }
                ...

        foo_close(s)
                xAssert(s->rcnt == 0);
                move sessn from active map to cached map

        foo_destroy(s)
                xDestroy(s);

\end{verbatim}

\end{itemize}


\subsection{Modifying v3.1 protocols}

In addition to the other modifications one needs to make to translate
an existing protocol from v3.1 to v3.2, one should make the following
modifications with respect to reference counts:

\begin{itemize}
\item{}
{\tt foo\_open} should not increment the reference count

v3.1 protocols often followed the format:

\begin{verbatim}
                foo_open()
                        if (session s exists) {
                                s->rcnt++;
                        } else {
                                s = xCreateSessn();
                                fill in state;                        
                        }
                        return s;
\end{verbatim}

        \xOpen{} now automatically increments the reference count so the
        format should be more like:

\begin{verbatim}
                foo_open()
                        if (session s does not exist) {
                                s = xCreateSessn();
                                fill in state;                        
                        }
                        return s;
\end{verbatim}        

\item{}
{\tt foo\_close} should not decrement the reference count
        
        In v3.1, {\tt foo\_close} was called whenever \xClose{} was called, so
        {\tt foo\_close} often looked like:

\begin{verbatim}
                foo_close()
                        if (--s->rcnt == 0) {
                                xDestroy(s);
                        }
\end{verbatim}
        
        In v3.2, the protocol's close function is called only when the
        reference count reaches zero (see the examples in the caching section.)


\item{}
\xPop{} should not be bypassed

        Since \xPop{} in v3.1 did nothing more than call the session's
        pop operation, v3.1 protocols often either called their pop
        routine directly or, if their pop routine was small, they
        absorbed this functionality into their demux routine and
        bypassed \xPop{} altogether.  Since \xPop{} 
        manipulates reference counts in v3.2, it
        can not be bypassed.


\item{}
\xDemux{} should not use \xOpen{} to create new sessions

It is common practice for a protocol to create passive sessions
(where session creation is triggered by an incoming packet rather than
an xOpen) by having {\tt foo\_demux} call {\tt foo\_open} and then
possibly customizing the session.
There is nothing wrong with this practice, but such a
protocol  should invoke {\tt foo\_open} directly rather
than using \xOpen{}.  The passively created session should have an
initial reference count of 0, but \xOpen{} would return a session
with an initial reference count of 1.

\end{itemize}


