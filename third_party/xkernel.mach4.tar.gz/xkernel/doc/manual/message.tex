% 
% message.tex
%
% $Revision: 1.20 $
% $Date: 1993/11/23 23:08:55 $
%

\section{Message Library}

The message library provides a set of efficient, high-level operations
for manipulating messages.  The underlying data structure that
implements messages is optimized for fragmentation/reassembly, and for
adding/stripping headers. 
Protocol
programmers should manipulate messages only with the operations
documented here.

Future releases of the message library will accomodate
the use of buffers that
are owned and managed by different buffer pools, including device
memory, user buffers, simple kernel buffers, and structured kernel
buffers such as BSD {\em mbufs} and System V {\em mblocks}.  

\subsection{Type Definitions}

Loosely speaking, Version 3.2 messages include two main components: a
{\it stack} that holds the message header, and pointer ({\it tree}) to
a tree structure of buffers that hold the data. 


\begin{tabbing}
xxxx \= xxxxxxxx \= xxxxxxxxxxx \= xxxxxxxxxxxx \= \kill
\>{\sem typedef struct} \{\\
\>\>  {\sem char}       \>*headPtr;\\
\>\>  {\sem char}       \>*tailPtr;\\
\>\>  {\sem char}       \>*stackHeadPtr;\\
\>\>  {\sem char}       \>*stackTailPtr;\\
\>\>  {\sem struct} \{\\
\>\>\>    {\sem unsigned short} \> numNodes;\\
\>\>\>    {\sem unsigned int}   \> myStack;\\
\>\>  \} state;\\
\>\>  {\sem MNodeLeaf}* \>stack;\\
\>\>  {\sem MNode}*     \>tree;\\
\>\>  {\sem void}*	\>attr;\\
\>\>  {\sem int}	\>attrLen;\\
\>\>  {\sem MNodeLeaf}* \>lastStack;\\
\>\>  {\sem char}       \>*lastStackTailPtr;\\
\>\>  {\sem struct} \{\\
\>\>\>    {\sem unsigned int}   \> myLastStack;\\
\>\>  \} tailstate;\\
\>\} {\bold Msg} ;
\end{tabbing}

\subsection{Constructor/Destructor Operations}

These operations are used to create and destroy messages. Many of
them are, for example, used by device drivers and system call code
that has to incorporate a data buffer into an \xk{} message.

Messages that are newly created ``own'' the stack space and can
write into that space efficiently using ``msgPush''.  See the section
on Usage (\ref{msgusage}) for more information about stacks.

\subsubsection{msgConstructEmpty}

Initializes a message structure with a data length set to zero.
The user must provide a pointer to valid memory.
\medskip

{\sem void} {\bold msgConstructEmpty} ({\sem Msg*} {\caps this});

\subsubsection{msgConstructBuffer}

Copies data from a user buffer into an uninitialized message
structure.  The message data area is allocated and a copy is
performed.  This constructor is used when the data buffer already
exists. Use {\em msgConstructAllocate} when you have the opportunity
to fill the buffer after it has been created.
\medskip

{\sem void} {\bold msgConstructBuffer} ({\sem Msg*} {\caps this},
{\sem char} *{\caps buf}, {\sem long} {\caps len});

\subsubsection{msgConstructAllocate}

Allocates 
a data area of size {\em len} and associates the area with the
uninitialized message structure {\em this}.  A pointer to the data
area is returned in {\em buf}.  A device driver might use this
constructor, handing {\em buf} to the device as a place to put down an
incoming packet.
\medskip

{\sem void} {\bold msgConstructAllocate} ({\sem Msg*} {\caps this}, {\sem long} {\caps len}, {\sem char} **{\caps buf});
\medskip

\subsubsection{msgConstructCopy}

The uninitialized message {\em this} will refer to the same data as
the message {\em another}.  No data is copied.  See also {\em
msgAssign}.

\medskip

{\sem void} {\bold msgConstructCopy} ({\sem Msg*} {\caps this}, {\sem Msg} *{\caps another});

\subsubsection{msgAssign}

The assignment of one message {\em m} to message {\em this} means that
message {\em this} will refer to the same data that as message {\em m}
currently does.  No data copying is involved.  This is equivalent to
doing a {\em msgDestroy} to {\em this}, followed by a {\em
msgConstructCopy}.  Therefore, this function should be used only when
both messages are valid.  Copying to an uninitialized structure should
be done with msgConstructCopy.
\medskip

{\sem void} {\bold msgAssign} ({\sem Msg*} {\caps this}, {\sem Msg*} {\caps m});
\medskip

\subsubsection{msgConstructInplace}

The uninitialized message {\em this} is constructed with a direct
reference to the buffer specified.  A function appropriate for freeing
the buffer when the message is destroyed must be provided.  The {\em
msgConstructInplace} function is recommended only for limited use
(e.g. within device drivers); it might not be compatible with later
implementations of the message library that will provide highly
efficient buffer management.

\medskip

{\sem void} {\bold msgConstructInplace} ({\sem Msg*} {\caps this},
	{\sem char} *{\caps buffer},
	{\sem long} {\caps length},
	{\sem Pfv} {\caps freefunc});

\subsubsection{msgConstructAppend}

The uninitialized message {\em this} is initialized to prepare for
appending data with {\em msgAppend}.  The message can be used in any
message operation, but {\em msgPush} will cause additional space to be
allocated.  A pointer to the beginning of the new message buffer is
returned in the {\em buffer} parameter.

\medskip

{\sem void} {\bold msgConstructAppend} ({\sem Msg*} {\caps this},
	{\sem long} {\caps length},
	{\sem char} **{\caps buffer})

\subsubsection{msgDestroy}

Logically frees message {\em this}. Data portions of the
message deallocated are freed as a result of this function
should there be no other outstanding references to them.
\medskip

{\sem void} {\bold msgDestroy} ({\sem Msg*} {\caps this});

\subsection{Manipulation Operations}

Protocols manipulate messages---e.g., add and strip headers, fragment
and reassemble packets---using the following set of operations.

\subsubsection{msgLen}

Returns the number of bytes of data in message {\em this}.
\medskip

{\sem long} {\bold msgLen} ({\sem Msg} *{\caps this});
\medskip

\subsubsection{msgTruncate}

Truncates the data in the message {\em this} to the given length.
Attempts to to reduce the length to less than zero will result in no
change to the message.  No storage is freed as a result of truncation.
This operation is used to strip trailers from a message.
\medskip

{\sem void} {\bold msgTruncate} ({\sem Msg*} {\caps this}, {\sem long} {\caps newLength});
\medskip

\subsubsection{msgChopOff}

Removes data from the message {\em this} and assign it to the message
{\em head}.  No copying is done. This operation is used to fragment
a message into smaller pieces.  Both messages must be valid at the
time of the call.
\medskip

{\em 3.1 Compatibility notes}:
The 3.1 version of this was {\em msg\_break}, but the arguments were reversed.
\medskip

{\sem void} {\bold msgChopOff} ({\sem Msg*} {\caps this}, {\sem Msg*} {\caps head}, {\sem long} {\caps len});
\medskip

\subsubsection{msgJoin}

Assigns (in the same sense as {\em msgAssign}) 
to {\em this} message the concatenation of messages {\em m1}
to the front of {\em m2}. This operation is used to reassemble fragments
into a larger message.  The first argument must be a valid message.  The
arguments need not refer to distinct messages.
\medskip

{\sem void} {\bold msgJoin} ({\sem Msg*} {\caps this}, {\sem Msg*} {\caps m1}, {\sem Msg*} {\caps m2});
\medskip

Example of using non-unique arguments:
\medskip
{\bold msgJoin} ({\caps this},  {\caps this}, {\caps MessageTail});
\medskip

\subsubsection{msgPush}

Prepends data pointed to by {\em hdr} to the front of message {\em
this}.  The supplied function {\em store} will be called with the {\em
hdr} pointer, a pointer to the message data area where the header data
can be copied, the {\em hdrLen} value indicating the length of the
data, and the {\em arg} pointer.  The last argument is not interpreted
by the {\em msgPush} function; its semantics are determined entirely
by the user store function.
\medskip

{\sem void} {\bold msgPush} ({\sem Msg*} {\caps this}, {\sem MStoreFun} {\caps  store}, {\sem void} *{\caps hdr}, {\sem long} {\caps hdrLen}, {\sem void} *{\caps arg});
\medskip

Function {\em store} typically converts the header to network byte
order and stores it in a potentially unaligned buffer; {\em len} is
the number of bytes to be read and {\em arg} is an arbitrary argument
passed through from {\em msgPush}.
\medskip

{\sem typedef void} ({\bold *MStoreFun} )({\sem void} *{\caps hdr}, {\sem char} *{\caps des}, {\sem long} {\caps len}, {\sem void} *{\caps arg});
\medskip

{\sanss Example Declaration: }{\sem void} {\bold copypush}({\sem void}
*{\caps src}, {\sem char} *{\caps dst}, {\sem long} {\caps  count}, {\sem void} *{\caps type}))

\subsubsection{msgPop}

Removes data from the front of message {\em this}.  After making the
requested data contiguous, {\em msgPop} calls the function {\em load}
with the {\em hdr} argument, a pointer to the start of the contiguous
data, {\em hdrLen}, and the {\em arg} pointer.  Function {\em load}
should return an integer representing the amount data that should be
popped off the message.  In optimized code, no validity check is done
on this number. This function returns TRUE if successful, FALSE if the
message is shorter than {\em hdrLen} bytes.
\medskip

{\sem bool}  {\bold msgPop} ({\sem Msg*} {\caps this}, {\sem MLoadFun} {\caps load}, {\sem void} *{\caps hdr}, {\sem long} {\caps hdrLen}, {\sem void} *{\caps arg});
\medskip

Function {\em load} loads the header from a potentially unaligned
buffer and converts it to machine byte order.  The {\em len} is the
number of bytes guaranteed contiguous and {\em arg} is an arbitrary
argument passed through from {\em msgPop}.  This function should
return the number of bytes to be actually popped from the message.
This information is used by {\em msgPop} to update the internal
message structure.
\medskip

{\sem typedef long} ({\bold *MLoadFun} )({\sem void} *{\caps hdr}, {\sem char} *{\caps src}, {\sem long} {\caps len}, {\sem void} *{\caps arg});
\medskip

{\sanss Example declaration: }{\sem long} {\bold copypop}(
{\sem void} *{\caps dst}, 
{\sem char} *{\caps src}, 
{\sem long} {\caps  count}, 
{\sem void} *{\caps type}))

\subsubsection{msgPopDiscard}

Discards {\em len} bytes from the front of message {\em this}.
Returns {\em TRUE} if successful.
\medskip

{\sem bool}  {\bold msgPopDiscard} ({\sem Msg} *{\caps this}, {\sem
long} {\caps len})
\medskip

\subsubsection{msgAppend}

Appends data pointed to by {\em tail} to the end of message {\em
this}.  The supplied function {\em store} will be called with the {\em
tail} pointer, a pointer to the message data area where the data
can be copied, the {\em tailLen} value indicating the length in bytes
fo the data, and the {\em arg} pointer.
If it is necessary to allocate new space to hold the data, the
{\em newlength} parameter designates the minimum amount of space 
for the allocation.  See {\em msgConstructAppend}, also.
\medskip

{\sem void} {\bold msgAppend} ({\sem Msg*} {\caps this}, {\sem
MStoreFun} {\caps  store}, {\sem void} *{\caps tail},
{\sem long} {\caps tailLen}, {\sem void} *{\caps arg},
{\sem long} {\caps newlength});
\medskip

\subsubsection{msgSetAttr}

Associates an {\em attribute} of {\em len} bytes with {\em name}
and attaches it to message {\em this}.  Setting an attribute overrides
any previous attribute with the same name.  Message attributes are used
to communicate ancillary properties of messages from a protocol to a session,
or between protocols.

The only name supported at this time is {\em 0}.  Attempting to set an
attribute with another name will result in an {\em XK\_FAILURE} return
value.  

\medskip

{\sem xkern\_return\_t} {\bold msgSetAttr} 
({\sem Msg} *{\caps this}, {\sem int} {\caps name},
{\sem void} *{\caps attribute}, {\sem int} {\caps len});
\medskip

\subsubsection{msgGetAttr}

Retrieves an attribute previously attached to message {\em this} with
{\em name}.  If no attribute has been associated with {\em name}, 
{\em 0} will be returned.
\medskip

{\sem void *} {\bold msgGetAttr} 
({\sem Msg} *{\caps this}, {\sem int} {\caps name} );

\subsubsection{msgForEach}

For each contiguous data area in the message, invokes the function {\em
f} with three arguments: the address of the data, the length of the
data, and a user supplied argument {\em arg}.  The iteration stops
when {\em f} does not return {\em TRUE}.
\medskip

{\sem void} {\bold msgForEach} ({\sem Msg} *{\caps this}, {\sem XCharFun} {\caps f}, {\sem void} *{\caps arg});
\medskip

{\sem typedef} {\sem  bool}  ({\bold *XCharFun} )({\sem char} *{\caps ptr}, {\sem long} {\caps len}, {\sem void} *{\caps arg});\\

\subsection{Usage Rules}
\label{msgusage}

The $x$-kernel coding conventions dictate that messages should be
destroyed by the same entity that originally constructed them.  Thus,
the ethernet driver is responsible for destroying messages after
successfully delivering them upward, and the top-level protocols that
interface to user functions should destroy messages that have been
successfully delivered to their destination.

When a protocol passes
a message to an adjacent protocol (via {\em xPush}, {\em
xDemux}, etc.) its view of the message becomes invalid.
The contents of the message after such an operation depend on
which lower headers were pushed onto it.  Should a protocol want to
keep a reference to the message---e.g., so it can later retransmit
it---it must explicitly save a copy using the {\em msgAssign} or {\em
msgConstructCopy} operation before passing the message on to another
protocol.

Note that although a protocol which constructs a message invalidates
its view of the message by performing a UPI operation involving
that message, it is still responsible for destroying the message. 

The stack ownership is a hidden variable in the message library
implementation that affects whether or not storage is automatically
allocated on {\em msgPush} operations.  The stack ownership is
affected by several message library operations, particularly
{\em msgAssign, msgJoin, msgPop}, and {\em msgConstructCopy}.
The user is referred to the source code for the details of the
ownership rules.

Message attributes passed between protocols should consist of
exportable data, i.e. not pointers.  Adherence to this convention will
ensure that the protocol can be in used in a multi-adress space
environment.
