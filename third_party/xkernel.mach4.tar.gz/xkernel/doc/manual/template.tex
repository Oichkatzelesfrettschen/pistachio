% 
% $RCSfile: template.tex,v $
%
% $Revision: 1.4 $
% $Date: 1993/11/18 06:05:33 $
%

\subsection*{NAME}

\noindent Name of the protocol. This name, given in all lower-case letters,
can be given as an argument to {\sanss xGetProtlByName} to get a capability
for the protocol. Note that there are multiple implementations of various
protocols; i.e., a given name might map to multiple implementations. The
implementation bound to a name in a given kernel is set in {\tt graph.comp}.

\subsection*{SPECIFICATION}

\noindent Reference to a document that gives the specification for the
protocol. In cases where no formal specification exists, this section
gives a high-level description of the protocol.

\subsection*{SYNOPSIS}

\noindent A brief description of what the protocol does. Outlines
any unusual features and bugs, including any features of the
protocol specification not implemented.

\subsection*{REALM}

\noindent
Indicates whether the protocol is in the ASYNC realm (supporting
push, demux and pop), the RPC realm (supporting call,
calldemux and callpop), the CONTROL realm (existing only to allow
control operations), or the ANCHOR realm (interfacing with the host
system.) 

\subsection*{PARTICIPANTS}

\noindent 
A discussion of the number of participants the
protocol expects to see and what it expects to see on the
participants' stacks.

\subsection*{CONTROL OPERATIONS}

\noindent 
Non-standard control operations supported by the protocol.  For each
control operation, the type of the input and output argument is given
(i.e., the type used to interpret the buffer argument) and whether the
control operation is understood by {\tt x\_controlsessn} or {\tt
controlprotl} (or both) is specified. In the case of control
operations that take multiple arguments, we give a set of types.
Non-primitive types are generally defined in the protocol's {\tt .h}
file.

\subsection*{EXTERNAL INTERFACE}

Interfaces not encapsulated within \xk{} operations

\subsection*{CONFIGURATION}

\noindent 
A description of configuration options for the protocol, including
descriptions of 
of what this protocol expects of the protocols below it.
If the protocol can only be configured above a certain protocol, the
appropriate graph.comp line is given explicitly.

\subsection*{AUTHORS}

\noindent Who to complain to if the protocol fails to work as advertized.
