% $RCSfile: crypt.tex,v $
%
% $Revision: 1.9 $
% $Date: 1993/11/09 00:18:11 $
%

\subsection*{NAME}
\label{CRYPT}

\noindent CRYPT and TCRYPT (DES Encryption Protocols)

\input cryptDist

\subsection*{SPECIFICATION}

\noindent
CRYPT encrypts all bytes of all packets sent using the DES encryption 
algorithm in cipher-block-chaining mode.
CRYPT uses the key manager protocol (KM, page~\pageref{KM}) to map from the 
source address passed in at open time to a DES key. Each message is 
encrypted independently. CRYPT is designed to be composed over any
datagram protocol, and it accepts arbitrary address types at 
open time.  CRYPT lengthens each packet by 1 to 8 bytes, to the
next larger multiple of 8 bytes.

TCRYPT is the same as CRYPT, except that packet lengths are usually unchanged.
However, TCRYPT alters packets in the size range 1 to 7 bytes.  All of
the information below for CRYPT also applies to TCRYPT, except for
packet sizes.

\subsection*{SYNOPSIS}

\noindent 
When a CRYPT session is opened, it opens the protocol configured below 
it with the addresses passed to it during open. It then performs a 
GETPARTICPANTS operation on the newly opened session 
to acquire the fully resolved addresses of the participants. The
set of participants is then used as the argument to open a key manager (KM)
session. 

When a message is pushed to a CRYPT session a KM\_RESOLVE operation is
done on the opened key manger session to get the current local key.
Because DES operates on blocks of 8 bytes, the message is zero-padded to
the next larger multiple of 8 bytes; the last byte is set to the
number of pad bytes added.  Total padding is at least 1 and at most 8
bytes.  The receiver uses KM\_RRESOLVE to ask his key manager for the
remote key.  He decrypts the message and removes the padding. If the
pad bytes are not 0, or the last byte has an out-of-range value for
the amount of padding, the message is dropped.  Assuming the incoming
length is a multiple of 8 bytes, the probability that a message
decrypted with a bad key will pass the length and pad checks is about
1/255.

TCRYPT preserves packet lengths, except those in range of 1-7 bytes.
For this reason, TCRYPT is not expected to be used below protocols
that produce very short messages.  The first part of the packet
encryption algorithm is the same as for CRYPT: as much of the packet
as possible is encrypted in blocks of eight bytes.  If there are
leftover bytes at the end of the packet, enough (already encrypted)
adjacent bytes from the packet are used to make up a complete block of
eight bytes, which is encrypted.  Decryption is straightforward.
Because the map is length preserving, there is no possibility of
decryption failing (although it might produce garbage).  CRYPT and
TCRYPT have each been tested on both big-endian and little-endian
machines, and between machines of different endianness.

CRYPT discards the attributes attached to any outgoing message.
This is done to minimize out-of-band information transmittal.
CRYPT does not touch any attributes attached to incoming 
messages, and forwards them along with the decrypted data up to the next 
protocol.

On outgoing messages, if the key manager should return a key of all 0s,
indicating it does not have the real key, the message is dropped.
For incoming messages, a key of all 0s is allowed.

\subsection*{REALM}

CRYPT is in the ASYNC realm.


\subsection*{PARTICIPANTS}

CRYPT passes participants to the lower protocols without manipulating
them.  It uses the participants to lookup the keys for encryption and
decryption.

\subsection*{CONTROL OPERATIONS}

CRYPT supports 16 control operations. 8 of these operations are 
passed to the incomming key manager and 8 are passed to the 
out going key manager. Every KM control operation as two 
CRYPT control operation. CRYPT\_IN\_SET sets the key on the 
incoming key manager by performing a KM\_SET operation on 
it. Likewise CRYPT\_IN\_SET peforms a KM\_SET on the outgoing 
key manager. The CRYPT control operations are listed below:  

\begin{verbatim}
CRYPT_IN_RESOLVE       
CRYPT_IN_SET          
CRYPT_IN_INVALIDATE   
CRYPT_IN_ISEMPTY      
CRYPT_IN_ISVALID      
CRYPT_IN_ISINVALID    
CRYPT_IN_KEYSIZE      
CRYPT_IN_HOSTSIZE     
CRYPT_OUT_RESOLVE       
CRYPT_OUT_SET          
CRYPT_OUT_INVALIDATE   
CRYPT_OUT_ISEMPTY      
CRYPT_OUT_ISVALID      
CRYPT_OUT_ISINVALID    
CRYPT_OUT_KEYSIZE      
CRYPT_OUT_HOSTSIZE     
\end{verbatim}


CRYPT diminishes the results of GETOPTPACKET 
and GETMAXPACKET to the next lower size congruent to 7 (mod 8);
this will reduce the size by at least 1 and at most 8 bytes.
TCRYPT does not change the packet sizes, except that values less than 8
are set to 0. 

CRYPT forwards all other control operations to the transport protocol 
configured below it. Due to security concerns CRYPT zero's out the 
buffer of any control it passes down the graph.  

\subsection*{CONFIGURATION}

CRYPT/TCRYPT expect to be configured on top of a transport protocol and a
key manager. The transport protocol must preserve packet boundaries
(i.e. CRYPT/TCRYPT will not work on top of TCP).
Because it changes packet lengths, CRYPT will only work between
IP and (UDP or TCP) if checksumming is disabled; TCRYPT is acceptable here.
Any protocol configured above TCRYPT should not generate
packets of length between 1 and 7 inclusive, because the length will
not be preserved.

Example of a graph.comp file:
\begin{verbatim}
---------------------------------
@;
name=simeth/0;
name=eth protocols=simeth/0;
name=arp protocols=eth;
name=vnet protocols=eth,arp;
name=ip protocols=vnet;
name=km trace=TR_ERRORS;
name=crypt protocols=ip,km trace=TR_ERRORS;
name=udp protocols=crypt;
name=udpcrypttest protocols=udp;
@;
prottbl = DEFAULT;
---------------------------------
\end{verbatim}

\subsection*{SHORTCOMINGS, THINGS TO BE DONE}

This version of CRYPT/TCRYPT is experimental, and some software
engineering and cryptographic changes have yet to be implemented.  The
protocol does not zero out message plaintext after encryption, or
ciphertext after decryption.  Keys are not zeroed out when the
sessions are closed.  There is a small storage leak when a session is
closed because the key memory is not returned.  The
cipher-block-chaining algorithm is always reinitialized, for every
message, with a starting vector of 0s, rather than random salt.  There
is no inter-message cryptographic chaining.  Our current system of one
key per host is suitable for testing, but would not be appropriate for
a real application.

The information passed during opening of a session includes
arbitrary participants; this provides an outbound information
channel for Trojan Horse client programs.  The lengths of outgoing
packets are another channel.  The presence or absence of communication
is a third channel.  A Trojan Horse client can also originate known
plaintext to help out a remote attacker.

The source and destination IP addresses appear in clear for every
packet on the net, so traffic analysis is trivial for a properly
located eavesdropper.  Message length is also available to the
eavesdropper.  Many protocols have standard length messages they send
(acknowledgements, routing updates, etc.), often with fixed, known, or
easily guessable contents.  These are also available to an opponent.

\subsection*{AUTHORS}

\noindent Richard Schroeppel, David C. Schwartz, and  Sean O'Malley
