%
% ssr-test.tex
%
% $Revision: 1.1 $
% $Date: 1993/02/01 22:33:16 $
%

\subsection*{NAME}

\noindent SSR test (tests port transfers with MachNetIPC and SSR)

\subsection*{SYNOPSIS}

\noindent 

\subsection*{REALM}

SSR test is not a protocol; it is specific to the Mach3 platform.

\subsection*{CONFIGURATION}

SSR test communicates with MachNetIPC and SSR.  There must be an \xk{}
task with MachNetIPC running on the host, and there must be a 
Mach nameserver running on the host.

SSR test takes command line arguments.  The first argument specifies
the test type.  The simplest test is the {\em postcard} test.  Other
test types are documented in the source code.

To start a postcard server at debugging level 1:

\noindent {\tt ssr-test -t p -l 1 \& };

To start a postcard client at debugging level 1, communicating with a
server at IP address 192.12.69.211, using 10 round-trips:

\noindent {\tt ssr-test -t p -p 192.12.69.211 -l 1 \& };

\subsection*{AUTHORS}

\noindent Hilarie Orman
