%
% tcp.tex
%
% $Revision: 1.6 $
% $Date: 1993/02/04 08:18:19 $
%

\subsection*{NAME}

\noindent TCP (Transmission Control Protocol)

\subsection*{SPECIFICATION}

\noindent {\it Transmission Control Protocol}. Request for Comments
793, USC Information Sciences Institute, Marina Del Rey, Calif., Sept. 1981

\subsection*{SYNOPSIS}

\noindent TCP is a reliable stream transport protocol. It maintains
a connection between the server and the client, and provides reliable
stream delivery to the process. This implementation is an encapsulation
of the Unix 4.3 BSD implementation.

This implementation of TCP supports input and output buffering.
Output buffers are contained within TCP.  If the amount of data sent
and unacknowledged by the peer reaches the output buffer size, TCP
will block subsequent {\em xPush}'s (or will return
XMSG\_ERR\_WOULDBLOCK in the case of non-blocking I/O.)

TCP provides support for users to work with finite input buffers.  TCP
will limit the amount of input data sent to its upper protocol via
{\em xDemux} to the size of the input buffer.  When data have been
consumed from the user's input buffer, free buffer space must be
signalled to TCP via a TCP\_SETRCVBUFSPACE call (see below.)  If a
user does not wish to use input buffering, a control message
signalling an empty buffer should be sent in response to each {\em
xDemux.} 


\subsection*{REALM}

TCP is in the ASYNC realm.

\subsection*{PARTICIPANTS}

TCP removes a pointer to a long (the TCP port number) from the
participant stack.  TCP ports must be less than 0x10000.  
If the local participant is missing, or if the
local protocol number is ANY\_PROT, TCP will select an unused local
port. 


\subsection*{CONTROL OPERATIONS}

\begin{description}

\item[{\tt TCP\_PUSH:}]
Force a TCP message to be sent.  (session only)
\begin{description}
\item[{\rm Input:}] none
\item[{\rm Output:}] none
\end{description}

\item[{\tt TCP\_GETSTATEINFO:}]
Returns state of the connection.  (session only)
\begin{description}
\item[{\rm Input:}] none
\item[{\rm Output:}] {\tt int}
\end{description}

\item[{\tt TCP\_DUMPSTATEINFO:}]
Prints out statistics gathered by TCP.  (protocol only)
\begin{description}
\item[{\rm Input:}] none
\item[{\rm Output:}] none
\end{description}

\item[{\tt TCP\_GETFREEPROTNUM:}]
Returns an unused TCP port number.  This port number will not be given
out to subsequent TCP\_GETFREEPROTNUM calls until it is released with
TCP\_RELEASEPROTNUM.  This allows an opener to separate reservation of
free ports from the actual open operation, if desired.
(protocol only) 
\begin{description}
\item[{\rm Input:}] none
\item[{\rm Output:}] long
\end{description}

\item[{\tt TCP\_RELEASEPROTNUM:}]
Releases a TCP portnumber previously acquired with TCP\_GETFREEPROTNUM.
(protocol only)
\begin{description}
\item[{\rm Input:}] long
\item[{\rm Output:}] none
\end{description}


\item[{\tt TCP\_SETRCVBUFSPACE:}]
Tells TCP how many bytes in the receive queue are free.
(session only)
\begin{description}
\item[{\rm Input:}] u\_short
\item[{\rm Output:}] none
\end{description}


\item[{\tt TCP\_SETRCVBUFSIZE:}]
Tells TCP the size of the TCP user's receive queue.
(session only)
\begin{description}
\item[{\rm Input:}] u\_short
\item[{\rm Output:}] none
\end{description}


\item[{\tt TCP\_GETSNDBUFSPACE:}]
Asks TCP for the number of free bytes its send queue.
(session only)
\begin{description}
\item[{\rm Input:}] none
\item[{\rm Output:}] u\_short
\end{description}

\item[{\tt TCP\_SETSNDBUFSIZE:}]
Tells TCP to change its send queue to the indicated size
(session only)
\begin{description}
\item[{\rm Input:}] u\_short
\item[{\rm Output:}] none
\end{description}


\item[{\tt TCP\_SETOOBINLINE:}]
Tells TCP whether users wants urgent data to be delivered inline
(non-zero == yes.)  (session only)

\begin{description}

\item[{\rm Input:}] int
\item[{\rm Output:}] none

\end{description}

\item[{\tt TCP\_GETOOBDATA:}]
reads the urgent data (exactly one byte), returning 1 on a successful
read or returning 0
if data was either read already or was not received yet
(the OOB notification may precede the actual reception
of the OOB data)
\begin{description}
\item[{\rm Input:}] none
\item[{\rm Output:}] char
\end{description}


\item[{\tt TCP\_OOBPUSH:}]
send a msg in urgent mode
\begin{description}
\item[{\rm Input:}] Msg *
\item[{\rm Output:}] char
\end{description}



\item[{\tt TCP\_OOBMODE:}]
TCP uses this to tell the user of TCP that it has
urgent data present, i.e., TCP does an xControl() call on
its parent --- THIS IS AN UPCALL!
The first void pointer (args[0]) is of type XObj and is
a pointer to the TCP session that invoked this operation.
The second pointer (args[1]) is of type u\_int and is the
value of the urgent data mark. 
The oobmark indicates that the "oobmark-th" byte in the
receive queue is the oobdata (or will be the oobdata.)

Note: all protocols using TCP without having OOB data delivered
in-band must be prepared to accept this upcall.


\begin{description}
\item[{\rm Input:}] void *args[2]
\item[{\rm Output:}] none
\end{description}


\end{description}

\subsection*{CONFIGURATION}

\noindent {\tt name=tcp protocols=ip;}

\subsection*{AUTHORS}

Norm Hutchinson, Herman Rao, and David Mosberger

