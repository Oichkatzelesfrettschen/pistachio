\section{Utility Routines}

\subsection{Storage: xMalloc and xFree}

These are just like Unix malloc and free functions.  The {\em xMalloc}
function will cause an $x$-kernel abort if no storage is available;
therefore, it has no error return value.
\medskip

{\sem char} {\bold *xMalloc}({\sem int} {\caps size})\\
\medskip

{\sem int} {\bold xFree}({\sem char} *{\caps buf})

\subsection{Time}

The \xk{} uses the same time structure as Unix:

\begin{tabbing}
xxxx \= xxxxxxxx \= xxxxxxxxxxx \= \kill
\>{\sem typedef struct} \{\\
\>\>  {\sem long}  sec;\\
\>\>  {\sem long}  usec;\\
\>\} {\bold XTime} ;\\
\end{tabbing}


\subsubsection{xGetTime}

Set {\em *t} to the current time of day:
\medskip

{\sem void} {\bold xGetTime} ({\sem XTime} *{\caps t})


\subsubsection{xAddTime}

Sets the {\em res} to the sum of {\em t1} and {\em t2.}
Assumes {\em t1} and {\em t2} are in standard time format (i.e., does not
check for integer overflow of the usec value.)
\medskip

{\sem void} {\bold xAddTime} 
({\sem XTime} *{\caps res}, {\sem XTime} {\caps t1}, {\sem XTime} {\caps t2})


\subsubsection{xSubTime}

Sets the {\em res} to the difference of {\em t1} and {\em t2.}
The resulting value may be negative.
\medskip

{\sem void} {\bold xSubTime} 
({\sem XTime} *{\caps res}, {\sem XTime} {\caps t1}, {\sem XTime} {\caps t2})
\medskip




\subsection{Byte Order: ntohs, ntohl, ntons, and htonl}

The byte order functions are the same as the Unix functions.
\medskip

\begin{tabbing}
xxxx \= xxxxxxxx \= xxxxxxxxxxx \= \kill
\>{\sem u\_short} {\bold ntohs}({\sem u\_short} {\caps n})\\
\>{\sem u\_long} {\bold ntohl}({\sem u\_long} {\caps n})\\
\>{\sem u\_short} {\bold htons}({\sem u\_short} {\caps n})\\
\>{\sem u\_long} {\bold htonl}({\sem u\_long} {\caps n})\\
\end{tabbing}

\subsection{Checksum}

\subsubsection{inCkSum}

Calculate a 16-bit 1's complement checksum over the buffer
{\em *buf} (of length {\em len}) and the msg {\em *m}, returning the bit
complement of the sum.  {\em len} should be even and the buffer must be
aligned on a 16-bit boundary.  {\em len} may be zero.
\medskip

{\sem u\_short} {\bold inCkSum} ({\sem Msg} *{\caps m}, {\sem u\_short} *{\caps buf}, {\sem int} {\caps len});

\subsubsection{ocsum}

Return the 1's complement sum of the {\em count} 16-bit words pointed
to by {\em hdr}, which must be aligned on a 16-bit boundary.
\medskip

{\sem u\_short} {\bold ocsum} ( {\sem u\_short} *{\caps hdr}, {\sem int} {\caps count});


\subsection{Strings to Hosts}

Utility routines exist for converting from string representations of
IP and Ethernet addresses to their structural counterparts and
vice-versa.  

\subsubsection{ipHostStr}

Returns a string with a ``dotted-decimal'' representation of IP host 
{\em h} (e.g., ``192.12.69.1'')  
This string is in a static buffer, so it must be copied if
its value is to be preserved.
\medskip

{\sem char *} {\bold ipHostStr} ( {\sem IPhost} *{\caps h} )


\subsubsection{str2ipHost}

Interprets {\em str} as a ``dotted-decimal'' 
representation of an IP host and assigns the fields of 
{\em h} accordingly.  The operation fails if {\em str} does not seem
to be in dotted-decimal form.
\medskip

{\sem xkern\_return\_t} {\bold str2ipHost} 
( {\sem IPhost} *{\caps h}, {\sem char } *{\caps str} )


\subsubsection{ethHostStr}

Returns a string with a representation of Ethernet host 
{\em h} (e.g., ``8:0:2b:ef:23:11'')  
This string is in a static buffer, so it must be copied if
its value is to be preserved.
\medskip

{\sem char *} {\bold ethHostStr} ( {\sem ETHhost} *{\caps h} )


\subsubsection{str2ethHost}

Interprets {\em str} as a six-hex-digit-colon-separated 
representation of an Ethernet host and assigns the fields of 
{\em h} accordingly.  The operation fails if {\em str} does not seem
to be in the correct format.
\medskip

{\sem xkern\_return\_t} {\bold str2ethHost} 
( {\sem ETHhost} *{\caps h}, {\sem char } *{\caps str} )

