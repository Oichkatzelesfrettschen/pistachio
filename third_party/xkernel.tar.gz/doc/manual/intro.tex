\section{Introduction}

%This draft document describes the $x$-kernel 3.2 interface, noting
%important differences from 3.1.  It is an evolving document that is
%intended to track the 3.2 interface closely during development.  The
%last modification was made on \today.

The $x$-kernel is an object-oriented multi-threaded architecture for
protocol implementation.  Each protocol is represented as an object
instance; sessions are specialized instances of protocol objects; each
thread is a calling sequence of object invocations.  Generally
speaking, each message in an $x$-kernel is processed by a thread of
protocol/session layers.  The $x$-kernel provides the interface
functions that allow one object to invoke another and subsystems for
manipulating messages, associative memory tables, events, addresses,
and debugging.

Version 3.2 of the \xk{} has changed significantly from version 3.1.
Unlike 3.1, which was a true {\it kernel} in the sense that it ran
stand-alone on Sun3 workstations, 3.2 is really a {\it protocol
implementation environment} that can be embedded in any platform,
including both hardware and software platforms. By defining all the
interfaces between the protocol and the host environment, the
\xk{} completely isolates the protocol from the underlying operating
system. As a result, protocol source code can be moved from one
platform to another without modification.

This manual covers the $x$-kernel functionality, the protocol library,
and the tools for building and running $x$-kernels on various
platforms. Sections 2--9 define the uniform protocol interface and the
various libraries that make up the \xk{}. Sections 10 and 11 give
additional guidelines about using these routines. Finally, Sections
12--17 describe procedures for maintaining, configuring, running, and
installing the \xk.

\subsection{Acknowledgements}

The $x$-kernel is the concept of Norman C. Hutchinson at the
University of British Columbia and Larry L.  Peterson of the
University of Arizona.  Sean W. O'Malley of the University of Arizona
has made major contributions.  The message library has received
attention from Peter Druschel.  Michael Pagels, Charles Turner, Mats
Bjorkman, and Richard Schroeppel have contributed to thread control
and device interfaces.  David Mosberger wrote the socket interface for
Mach.  Hilarie Orman has been responsible for several issues related
to Mach interfaces, and Ed Menze has remolded and improved almost
every aspect of the system.  Many people have contributed protocols
to the \xk{} as noted in appendix \ref{protman}.

\subsection{Our Address}

Please let us know of any problems you encounter so that we can
continue to improve the distribution. Our mail address is:

\begin{quote}
The $x$-Kernel Project\\
Department of Computer Science\\
University of Arizona\\
Tucson, AZ 85721
\end{quote}

\noindent We can be reached by electronic mail at:

\begin{quote}
xkernel-bugs@cs.arizona.edu\\
uunet!arizona!xkernel-bugs
\end{quote}

\noindent Because of limited resources we can't promise to fix every 
problem, but we appreciate all comments.

Also, we typically post messages about the \xk{} (including notices of
future releases) to xkernel-interest@cs.arizona.edu. Send mail to
xkernel-interest-request@cs.arizona.edu to be added to to this mailing
list.

\subsection{Copyright Notice}

\input copyright
