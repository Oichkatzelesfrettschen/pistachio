% 
% $RCSfile: install.tex,v $
%
% x-kernel v3.2
%
% Copyright (c) 1993,1991,1990  Arizona Board of Regents
%
%
% $Log: install.tex,v $
% Revision 1.14  1993/11/30  17:14:10  menze
% Added note about two different protocol tables
%

\section{System Installation}
\label{install}

This section describes how to install Version 3.2 of the $x$-kernel.
By installing the $x$-kernel we mean setting up the $x$-kernel
environment so that programmers can begin writing protocols and
configuring new kernels.

The $x$-kernel is commonly used for both research and instruction.  A
typical scenario is for a researcher (instructor) to install the
$x$-kernel for use by various network projects (students). Individual
projects (students) then build and test kernels in private working
directories. This section
describes the work the researcher (instructor) needs to do to set up
the $x$-kernel directory.  

\subsection{Unpacking the {\it tar} file}

If you haven't done so already, create a new directory for installing
the $x$-kernel.  For the purpose of this report, we refer to this
directory as {\tt /usr/xkernel}.  You may chose another name, however.
{\tt cd} to this directory and type

\begin{quote}
{\tt tar xf} {\it file}
\end{quote}

\noindent where {\it file} is the disk or tape file of the $x$-kernel
distribution. This will create several subdirectories in {\tt
/usr/xkernel}.  A description of the \xk{} directory hierarchy can be
found in the README file at the top level of the \xk{} hierarchy.


\subsection{About Compilers}

Version 3.2 of the $x$-kernel uses either the Gnu C compiler ({\tt
gcc}) or the standard Unix compiler ({\tt cc}).  If you do not
currently have Gnu C at your site, you can get a copy by doing
anonymous FTP to {\tt prep.ai.mit.edu}, and retrieving {\tt
pub/gnu/gcc.xtar}.

We recommend building the sunos version of the \xk{} with a gcc
earlier than 2.0, if possible.


\subsection{Building the Utility programs}

The \xk{} requires a few utility programs.  We distribute
binary versions of these programs, so you probably don't need to
rebuild these programs and can skip to the next section.

If you decide to rebuild the utility programs, 
you will need to define the environment variable {\tt ARCH} to
correspond to the type of binaries you want created: {\tt sparc}, {\tt
mips}, or {\tt intelx86}.

The \xk{} depends on gnumake, so you must first have gnumake built and
installed.  A copy of the source code and build instructions for
gnumake version 3.66 can be found in util/make.  After you have built
gnumake, place the binary in the appropriate /usr/xkernel/bin/ARCH
directory, where ARCH is as described above.

Next, put {\tt /usr/xkernel/bin/ARCH} and {\tt /usr/xkernel/bin}
in your search path.
They should appear before {\tt /bin} and {\tt /usr/bin} in
order to pick up gnumake before the standard Unix {\tt make}. Finally,
{\tt cd~/usr/xkernel/util} and type {\tt make~setup} followed by 
{\tt make~install}.  This will install a handful of programs used to
develop new kernels.  ({\tt make~setup} may produce a warning of the
form:  {\tt make[1]: fopen: Makedep.ARCH: No such file or directory}.
These missing files will be created during this step and the warning can
be ignored.)


\subsection{Building and testing the system}

As the first person installing the \xk{} at your site, you should look
over the makefiles in /usr/xkernel/build/Template.  They have many
site-specific parameters referring to compilers, libraries and the
like which should be set to reflect the environment at your site.

The \xk{} is distributed with two versions of the protocol number
table in {\sanss /usr/xkernel/etc}, {\sanss prottbl.std} and {\sanss
prottbl.nonstd}.  Since several example graph.comp and rom files refer
to a file named {\sanss prottbl} in that directory, you should
probably copy or link the version most appropriate for your site to
this name.  See section~\ref{protnum} for a description of the
difference between the two.

To install the system, build a kernel as described in section
\ref{config}, paying attention to the notes for system installers. 
You will want to configure a kernel that includes one or more of the
test protocols (e.g., {\tt udptest}).  You should then be able to run
your test kernel as described in section \ref{running}.

