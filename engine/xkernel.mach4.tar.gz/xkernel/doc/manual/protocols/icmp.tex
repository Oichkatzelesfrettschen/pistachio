%
% icmp.tex
%
% $Revision: 1.3 $
% $Date: 1992/02/11 01:37:18 $
%

\subsection*{NAME}

\noindent ICMP (Internet Control Message Protocol)

\subsection*{SPECIFICATION}

\noindent J. Postel. {\it Internet Protocol}. Request for Comments 
792, USC Information Sciences Institute, Marina del Ray, Calif., Sept. 1981.
;
\subsection*{SYNOPSIS}

\noindent ICMP handles control messages for IP. This implementation is
complete in that it handles all possible incoming ICMP requests.

\subsection*{REALM}

ICMP is in the CONTROL realm.  ICMP sessions may be opened to allow
control operations.

\subsection*{PARTICIPANTS}

ICMP neither removes nor adds anything to the participant stacks.  
It passes the participants directly to IP. 

\subsection*{CONTROL OPERATIONS}

\begin{description}

\item[{\tt ICMP\_ECHO\_REQ:}]
Send an ICMP Echo Request message to the peer host and wait for a
reply.  The buffer contains the length of the message.  Returns 0 if
successful, -1 if a timeout occurred.  (session only)
\begin{description}
\item[{\rm Input:}] {\tt int} 
\item[{\rm Output:}] none
\end{description}

\end{description}


\subsection*{CONFIGURATION}

\noindent {\tt name=icmp protocols=ip;}

\subsection*{AUTHOR}

\noindent Clinton Jeffery
