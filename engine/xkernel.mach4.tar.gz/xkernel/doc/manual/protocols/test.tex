% $RCSfile: test.tex,v $
%
% $Revision: 1.3 $
% $Date: 1993/11/17 19:01:39 $
%

\subsection*{NAME}

\noindent TEST (instantiated as 'chantest', 'udptest', etc.)

\subsection*{SPECIFICATION}

The test protocol runs a ``ping-pong'' test of the protocol below it
for various message lengths and numbers of iterations. 

\subsection*{SYNOPSIS}

Transport test protocols run in one of two roles, either as ``client''
or as ``server.''  The client will send a message to the server and
wait for a reply before sending the next message.  There are no
provisions for retransmission -- if the protocol below the test
protocol drops a message, the test will fail.


\subsection*{CONFIGURATION}

When the test protocol instantiates, it can determine which role it
should assume in several ways.  Command line parameters can be used to
cause the same kernel to run as the server on one machine and as the
client on another.  The server should be started up with a ``-s''
flag:

\begin{quote}
\begin{tt}
xkernel -s
\end{tt}
\end{quote}

The client side must be told the host address of the server peer (note
that on the sunos platform, this should be the address of the
simulated IP host.)  This can be done with the ``-c'' command line
option, e.g.:

\begin{quote}
\begin{tt}
xkernel -c192.12.69.54
\end{tt}
\end{quote}

If you will be running several tests between the same hosts you may
find it convenient to copy the test protocol to your build directory
and edit the declaration of ServerAddr and ClientAddr in your local
copy to name these hosts directly.  
This will eliminate the need for client and server flags.  Note
that most test protocols include other files from the {\sanss
protocols/test} directory, so you will either have to copy those files
as well or add {\sanss \$(XRT)/protocols/test} to the 
{\sanss TMP\_INCLUDES} variable in your Makefile.

If a test protocol is configured with an instance name of 'client' or
'server', it will come up in the appropriate role.  This can be used
to run both a client and server in the same kernel for loopback
testing as in this graph.comp excerpt:

\begin{verbatim}

...
name=udp protocols=ip;
name=udptest/server protocols=udp;
name=udptest/client protocols=udp;

\end{verbatim}

If you have configured several standard test protocols into the
kernel, you can run any subset of them by putting the test names onto
the command line, e.g.:

\begin{quote}
\begin{tt}
xkernel -testip -testudp
\end{tt}
\end{quote}

\noindent
With no command line test selections, all of the configured test protocols
will run.

The number of round trips for each packet size can be set with the
``trips'' flag:

\begin{quote}
\begin{tt}
xkernel -trips=10000
\end{tt}
\end{quote}


If you use ``dns'' lines (see \ref{dns}) in your rom file to map host
names to IP addresses, then you can use the name in place of the IP
address when starting the client, i.e.,

Rom file entry:
\begin{quote}
\begin{tt}
dns mars 192.12.69.54
\end{tt}
\end{quote}

Sample command lines for starting client:
\begin{quote}
\begin{tt}
xkernel -c mars
xkernel -cmars
\end{tt}
\end{quote}

\noindent
Note that the name on the command line must be an exact match (not a
substring) of the rom file entry.

The test protocols all use the common trace variable {\tt prottest}
which can be set in the third section of graph.comp:

\begin{quote}
\begin{verbatim}
@;
...
name=udptest	protocols=udp;
@;
name=prottest	trace=TR_EVENTS;
\end{verbatim}
\end{quote}


\noindent
If you set a
trace level when you declare the test protocol in the second section
of graph.comp, it will be ignored. 


\subsection*{CAVEATS}
Remember that if you are using {\em simeth} you must use the name of
the {\em simulated} host when you invoke the client, not the real host.
