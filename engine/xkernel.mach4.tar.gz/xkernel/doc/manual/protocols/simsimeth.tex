%
% simsimeth.tex
%
% $Revision: 1.4 $
% $Date: 1993/11/18 06:05:06 $
%

\subsection*{NAME}

\noindent SIMSIMETH ( Simulated Simulated Ethernet Driver Protocol)

\subsection*{SPECIFICATION}

\noindent
SIMSIMETH simulates an \xk{} ethernet driver by sending and receiving
messages using any x-kernel protocol that accepts UDP addresses. SIMSIMETH is platform independent. SIMSIMETH interoperates with SIMETH
( and itself). SIMSIMETH's primary purpose is to support the testing of 
protocols between x-kernel simulators such as the SunOS simulator 
and native mode x-kernel implementations such as the Mach 3.0 implementation. 


\subsection*{SYNOPSIS}

\noindent 
Each instantiation of SIMSIMETH is associates a {\em simulated} 
Ethernet address with a specific UDP address and simulates an 
Ethernet driver for a single interface.  
SIMSIMETH reads the UDP port from the ROM file and gets its IP
address by performing a GETMYHOST control operation on 
the protocol configured below it. 

The mapping between Unix UDP ports and SIMSIMETH Ethernet addresses is
very simple.  The six bytes of SIMSIMETH Ethernet address are formed by
the concatenation of the four byte IP host number for the Unix host on
which the simulator is running and the two byte UDP port used
by the SIMSIMETH instantiation.  Note that this IP host must be 
valid and defined by the protocol graph below SIMSIMETH. Since one 
generally runs two copies of ARP when SIMSIMETH is configured
keeping your  IP addresses straight is sometimes difficult. 
See the CONFIGURATION section below.

When a message is sent using SIMSIMETH a map is checked to see 
if a lower level session exists for that destination address. 
If no lower level  session exists one is created by performing an  
open on the protocol  configured below SIMSIMETH (most probably) UDP. 
Once SIMSIMETH has opened a lower level session it is never closed. 
SIMSIMETH then pushes an Ethernet header on the message 
and pushes the message using the lower level session.
When a packet arrives at SIMSIMETH the Ethernet header is removed 
and SIMSIMETH presents the packet received  as 
an incoming Ethernet packet.

Note that an \xk{} may be configured with multiple instantiations of
SIMSIMETH, each with its own UDP port, to simulate a multihomed host.
SIMSIMETH can awkwardly simulate Ethernet broadcast messages.  
When an
outgoing broadcast message is sent to SIMSIMETH, SIMSIMETH asks its
corresponding  ARP
protocol  for a dump of all hosts in its table.
SIMSIMETH then sends the message to each of these hosts in a
point-to-point fashion.  Note that for a reasonable simulation of
Ethernet broadcast, all
\xk{}s in communication should have the same ARP table (see the ARP
appendix.) 

The primary purpose of SIMSIMETH is to allow simulated x-kernels and 
native x-kernels to interoperate. This is possible because SIMSIMETH 
interoperates with the SIMETH protocol. Interoperability is achieved as 
follows. The graph.comp file for the simulated x-kernel is unchanged and 
is rooted at SIMETH the simulated Ethernet driver.  The graph.comp for the
machine running a native x-kernel must contain the same graph in the 
simulated version accept that SIMSIMETH is configured instead of SIMETH
and SIMSIMETH is configured on top of the standard Internet protocol 
graph.  If the following was the graph.comp on the SunOS simulator:

\begin{verbatim}
#SunOS Graph.comp
@;
name=simeth;
name=eth protocols=simeth;
name=arp protocols=eth; 
name=vnet protocols=eth,arp; 
name=ip protocols=vnet; 
name=udp protocols=ip;
name=udptest protocols=udp;
@;
prottbl = ../prottbl.simsimeth;
\end{verbatim}

The corresponding graph for the native mode x-kernel would be:

\begin{verbatim}
#Mach 3.0 Graph.comp
@;
name=ethdrv/SE0;
name=eth/lower protocols=ethdrv/SE0;
name=arp/lower protocols=eth/lower;
name=vnet/lower protocols=eth/lower,arp/lower;
name=ip/lower protocols=vnet/lower;
name=icmp/lower protocols=ip/lower;
name=udp/lower protocols=ip/lower;
name=simsimeth protocols=udp/lower;
name=eth/upper protocols=simsimeth;
name=arp/upper protocols=eth/upper;
name=vnet/upper protocols=eth/upper,arp/upper;
name=ip/upper protocols=vnet/upper;
name=udp/upper protocols=ip/upper;
name=udptest protocols=udp/upper;
@;
prottbl = ../prottbl.simsimeth;
\end{verbatim}

The /lower protocols in the above graph.comp correspond to the Internet 
protocol suite implemented in the SunOS kernel. The /upper protocols 
correspond the protocols running on the simulated x-kernel. Note that 
a user defined protocol table is needed. This is done because the 
protocol table must define the correct (800) relative ETH protocol id for 
IP for this  technique to work. The default protocol table defined by 
the x-kernel defines the IP protocol number as 3900 to avoid interference 
when testing new protocols. Note also that if you use SIMSIMETH 
please make sure that your versions of UDP, IP, ARP, VNET, and ETH
are at least as new as SIMSIMETH. 

There must be a similar correspondence between the ROM files on the 
SunOS side and the native Mach 3.0 side. For example the following 
is an example ROM file for the SunOS platform:

\begin{verbatim}
simeth 9875
eth mtu 1400
arp 192.12.69.1 192.12.69.67 9875
arp 192.12.69.2 192.12.69.99 1234
\end{verbatim}


While the following is an example ROM file for the Mach 3 
platform:

\begin{verbatim}
arp/lower     192.12.69.99    08:00:2b:23:6d:ec # mozart 
simsimeth 1234
eth/upper mtu 1400
arp/upper 192.12.69.1 c0:0c:45:43:26:93 # translation:  192.12.69.67 9875
arp/upper 192.12.69.2 c0:0c:45:63:04:d2 # translation:  192.12.69.99 1234
\end{verbatim}

The SunOS ROM file is identical to the standard ROM file except for the
addition of a line to set the Ethernet MTU to 1400 bytes. 
If the default MTU of 1500 bytes were to be used SunOS IP would 
fragment the outgoing simulated Ethernet packets into two 
real Ethernet packets of 1500 bytes and 64 bytes respectively. 
This pattern of packets can result in a serious increase in the number 
of dropped packets. The MTU of the simulated Ethernet driver 
on the Mach platform (eth/upper) is also set to 1400 bytes to 
avoid the same problem. Note you should always manually set 
the MTU's of the simulated Ethernet drivers to the same number!

The ROM for Mach 3.0 must be changed 
to set the appropriate IP to Ethernet address correspondence for the real
ARP (ARP/lower). While the simulated ARP (ARP/upper) must be configured 
with the same information as given in the SunOS ROM file. This is made
more complex because ARP on Mach 3.0 platforms expects real Ethernet 
addresses while the ARP on the SunOS platform expects to find simulated
Ethernet address. Therefore the user must manually convert a port 
IP address pair into an Ethernet address using the algorithm given 
above. For example the "Ethernet address": c0:0c:45:43:26:93 is 
simply the hex translation of 192.12.69.67 9875. 

Note that when the client is started it should be passed the simulated 
IP address of the server (192.12.69.{1or2}) not the real IP address 
of the server.

Working graph.comp, ROM, and protocol table files can be found in the 
Template directory. 

\subsection*{REALM}

SIMSIMETH is in the ASYNC realm, supporting the Ethernet driver interface
described in the ETH appendix.


\subsection*{PARTICIPANTS}

SIMSIMETH supports the Ethernet driver interface rather than a standard
xkernel UPI interface and thus makes no use of participant stacks.


\subsection*{CONTROL OPERATIONS}

\begin{description}

\item[{\tt ETH\_REGISTER\_ARP:}]
Used by an ARP instantiation to register itself with its corresponding
SIMSIMETH driver.  This is used to simulate Ethernet broadcasts as
described above.  If no ARP protocol registers with a SIMSIMETH
instantiation, broadcasts on that instantiation will not be
possible. 
\begin{description}
\item[{\rm Input:}] {\tt XObj /* ARP protocol object */ }
\item[{\rm Output:}] none
\end{description}

\end{description}


\subsection*{EXTERNAL INTERFACE}

SIMSIMETH supports the Ethernet driver interface
described in the ETH appendix.


\subsection*{CONFIGURATION}

SIMSIMETH should be configured on top of any protocol that takes 
UDP addresses and below any protocol which supports the ETH 
lower level driver interface.  It can be configured in either the
driver section or the protocol section of graph.comp. 


\medskip

\noindent
SIMSIMETH recognizes the following ROM options:

\smallskip

{\tt simsimeth nnnn}:
This instantiation of simsimeth should use UDP port nnnn.  There must be
such a line for each instantiation of SIMSIMETH in the \xk{}.


\subsection*{AUTHOR}

\noindent Sean O'Malley


