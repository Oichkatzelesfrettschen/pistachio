%
% ssr.tex
%
% $Revision: 1.6 $
% $Date: 1993/11/29 21:24:18 $
%

\subsection*{NAME}

\noindent SSR (a simple name server protocol for use with MachNetIPC).

\subsection*{SPECIFICATION}

See the file ``ssr.h''.  This contains the request identifier list
and the data structure defining an SSR message.  This is not MiG generated.

\subsection*{SYNOPSIS}

\noindent SSR is used for bootstrapping ports into MachNetIPC.
It can register a server port and forward requests to its peers 
on remote hosts. SSR listens for messages on its main port.
This port accepts Mach messages containing a service identifier, a request
identifier, an IP address, and a Mach reply port.  The requests are sent 
to the remote peer, and the Mach reply port is translated into a MachNetIPC
port.  The request is forwarded to the server registered on the remote
host (if any).  There is no reply to SSR.  

\subsection*{REALM}

SSR is in the anchor protocol realm; it is specific to the Mach3 platform.
SSR communicates with the NNS protocol using a globally defined Mach port; 
this requires that the two protocols exist in the same Mach task.

\subsection*{CONFIGURATION}

SSR communicates with MachNetIPC as its only lower protocol.

\noindent {\tt name=ssr protocols=machripc;}

\subsection*{AUTHOR}

\noindent Hilarie Orman
