% 
% $RCSfile: rom.tex,v $
%
% x-kernel v3.2
%
% Copyright (c) 1993,1991,1990  Arizona Board of Regents
%
%
% $Log: rom.tex,v $
% Revision 1.4  1993/11/30  17:25:42  menze
% Became a subsection
% Parsing library stuff moved to 'util' section
%


\subsection{ROM files}
\label{romfile}

To support run-time options for protocols and various subsystems, the
\xk{} provides the notion of a ROM file.  When a protocol instance or \xk{}
subsystem initializes, it typically scans a list of user-provided
options in the ROM file to see if it should adjust its default
parameters for that particular instantiation.  ROM options are used
for a variety of purposes, such as providing initial values for
databases, specifying numbers of network shepherd threads, and
providing IP gateway information.

Each ROM file entry consists of a single line.  The first field in
each line specifies the particular protocol or subsystem that
should interpret that line.  The rest of the fields are
specific to that particular protocol or subsystem.  Comments can be added
following a {\tt \#}.  As an example:

\begin{verbatim}

      #
      # Example ROM file
      #

      simeth 	port  1234

      arp 192.12.69.49  192.12.69.1 1234
      arp 192.12.69.45  192.12.69.1 9876

      prottbl /usr/xkernel/etc/prottbl.nonstd

\end{verbatim}


The SIMETH protocol will interpret the first line, the ARP protocol
will interpret the second and third lines and the protocol table
subsystem will interpret the last line.  

The exact method for providing ROM files is specific to the individual
platforms and is documented for each platform in section
\ref{running}. 

Protocols which provide ROM file configurable options will describe
the format of these options in their man pages in appendix
\ref{protman}. 


\input genrom.tex
