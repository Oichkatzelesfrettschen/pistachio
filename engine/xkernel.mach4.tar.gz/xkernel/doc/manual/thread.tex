% 
% thread.tex
%
% $Revision: 1.19 $
% $Date: 1993/11/20 03:11:07 $
%

\section{Thread Library}
\label{thread}

The \xk{} uses a ``thread-per-message'' model of computation, and
provides primitives for synchronizing threads.  The following
operations affect
thread scheduling.  Of these, only {\em semWait}
can cause the $x$-kernel to run a different thread than the current one.

Note that this section does not define any operations for creating or
destroying threads. This is because \xk{} threads are created and
destroyed implicitly: threads are created by the device driver (in the
case of incoming messages), by the system call interface (in the case
of outgoing messages), and by the event library (in the case of an
event firing). Threads are destroyed when they return from the
outer-most procedure.

\subsection{Type Definitions}

The only thread-related type that protocol programmers need be
aware of is the type {\em Semaphore}. However, this type is defined
by the underlying platform and is opaque to the protocol programmer.

\subsection{Synchronization Operations}

\subsubsection{semInit}

Initializes semaphore {\em s} with a count of {\em cnt}.  Semaphores
in the kernel are normally allocated statically (i.e., {\em Semaphore
x}) and must be initialized ({\em semInit(\&x, 1)}) before they are
used.\medskip

{\sem void} {\bold semInit} ({\sem Semaphore}  *{\caps s}, {\sem int} {\caps cnt})\\

\subsubsection{semWait}

Increments the use count for the semaphore.  The current thread will
either acquire the semaphore {\em s} or give up control until a {\em
semSignal} is done by another thread and the scheduler runs.
\medskip

{\sem void} {\bold semWait} ({\sem Semaphore}  *{\caps s})\\

\subsubsection{semSignal}

The current thread decrements the use count for semaphore {\em s}.
The current thread continues executing.  Note that if multiple threads
are blocked on the semaphore, there is no policy about which thread
will be awakened by the {\em semSignal.}
\medskip

{\sem void} {\bold semSignal} ({\sem Semaphore}  *{\caps s})


\subsection{Delay}

Delays the current thread for at least {\em msec} milliseconds.  
This is not a thread primitive, but a library routine built on top of
{\em semWait}/{\em semSignal}.
Note that the argument is in milliseconds, while the time argument to {\em
evSchedule} is in microseconds.
\medskip

{\sem void} {\bold Delay} ( {\sem int} {\caps} msec )


\subsection{Locking Operations}

The \xk{} provides a readers/writers lock synchronization type, {\em
ReadWriteLock}.  This type is currently built on top of the \xk{} {\em
Semaphore} type.  Although originally developed for the machNetIPC
protocols, the lock can be used for more general synchronization.  The
{\em ReadWriteLock} implementation favors writers (i.e., if queues of
readers and writers both exist, writers will always be favored.)

Note that we expect these locking operations to be used in a
multiprocessor environment, although we have not yet used them for
this purpose. Protocol developers working in an MP environment should
attempt to adhere to this interface. However, we consider locking on
an MP to be an unresolved issue, and so we are eager to hear about
experiences using these operations in such an environment.

\subsubsection{rwLockInit}

Initializes the lock for subsequent operations.  

\medskip
{\sem void} {\bold rwLockInit} ({\sem ReadWriteLock} *)


\subsubsection{rwLockDestroy}

All threads waiting on this lock will be released and their locking
operations will return {\em XK\_FAILURE}.  When both the reader and writer
queue are empty, the user-supplied {RwlDestroyFunc} will be called
with a pointer to the lock and the {\em void}* argument.

\medskip

{\sem void} {\bold rwLockDestroy} 
( {\sem ReadWriteLock} *, {\sem RwlDestroyFunc}, {\sem void} *)

\medskip
{\sem typedef void} ({\bold *RwlDestroyFunc} )
( {\sem ReadWriteLock} *, {\sem void} * );


\subsubsection{readerLock}

Acquires a reader lock.  Many readers may hold the lock at one time,
but no writers will acquire the lock while at least one reader holds
the lock.  Will return {\em XK\_FAILURE} if the lock was destroyed before
the lock was acquired.

\medskip
{\sem xkern\_return\_t} {\bold readerLock} ( {\sem ReadWriteLock} * )


\subsubsection{readerUnlock}

Releases a reader lock.  It is an error to call {\em readerUnlock}
without holding a reader lock.

\medskip
{\sem void} {\bold readerUnlock} ( {\sem ReadWriteLock} * )


\subsubsection{writerLock}

Acquires a writer lock.  While a writer lock is held, no other reader
or writer may acquire the lock.  
Will return {\em XK\_FAILURE} if the lock was destroyed before
the lock was acquired.

\medskip
{\sem xkern\_return\_t} {\bold writerLock} ( {\sem ReadWriteLock} * )


\subsubsection{writerUnlock}

Releases a writer lock.  It is an error to call {\em writerUnlock}
without holding a writer lock.

\medskip
{\sem void} {\bold writerUnlock} ( {\sem ReadWriteLock} * )



\subsection{Usage Rules}

\subsubsection{Scheduling and Preemption}

The currently executing thread gives up control by either terminating
or executing a {\em semWait} operation.  In other words, the \xk{}
does not preempt threads; threads voluntarily give up control of the
processor.  However, because each protocol is assumed to be an
independent component, protocols are written to assume that control
may be given up when a higher or lower level protocol is invoked.
Therefore, all protocol-protocol operations are considered to have the
potential to cause a thread switch, and all data structures must be
``secured'' before calling such operations.

\subsubsection{Blocking}

Although the \xk{} advocates a ``thread-per-message'' model, and it
provides primitives for blocking threads, as a general rule, threads
should not block except when waiting for a reply in an RPC-like
protocol. In most other cases, should a thread (message) not be able
to proceed, it should put the message in a protocol-dependent queue
and return. Later, another thread can pick the message up out of the
queue and continue processing it.

For example, when an incoming thread/message arrives in IP and
discovers that it is just one fragment of a larger datagram, rather
than blocking the thread and waiting for the other fragments to arrive,
the thread should insert the fragment into a reassembly buffer and
return. The thread that delivers the last fragment will then
reassemble the fragments into a single datagram and continue.

\subsubsection{External Threads}
\label{ext-threads}

Where the $x$-kernel is embedded in another operating system, there
may be asynchronous threads representing device drivers or user
requests that want to enter the \xk{}.  These threads must, in
general, acquire the \xk{} master lock (i.e., enter the \xk{} monitor)
with {\em xk\_master\_lock} before performing any \xk{} operations,
including other thread synchronization operations.  (This isn't
necessary for normal \xk{} threads because threads started by {\em
evSchedule} acquire the master lock automatically when they start
running.)  Unless a call is explicitly documented otherwise, threads
may not make \xk{} system or library calls without holding the master
lock.

A thread acquires and releases the master \xk{} lock with the following
operations.
\medskip

{\sem void} {\bold xk\_master\_lock} ( {\sem void} )

\medskip

{\sem void} {\bold xk\_master\_unlock} ( {\sem void} )

\bigskip

Note that normal protocols should not use these operations. The only
place that they are meaningful is in anchor protocols, such as device
drivers and application-level interface, that have to transition
between the \xk{} and the host OS. Also note that this interface is
not part of the official \xk{} interface; it is internal to the
current implementation of the \xk{}.

\subsubsection{Thread Turnaround}

Protocols should refrain from taking threads which are shepherding
outgoing messages down the protocol stack and turning them around to
accompany messages traveling up the protocol stack.  Since protocols
{\em are} allowed to reverse thread direction from incoming to
outgoing, allowing turnaround from outgoing to incoming could lead to
a thread caught in a recursive loop.  If an outgoing thread needs to
send a message back up, it should start a new thread to do this.  The
push routine of the ethernet protocol ({\sanss
/usr/xkernel/protocols/eth}) has an example of how this is done.

\subsubsection{Multiprocessor Support}

To date, we do not have significant experience running the \xk{} on a
multiprocessor. We have introduced locking operations that are useful
in certain situations on a uniprocessor, and while these operations
might extend to an MP environment, we cannot be certain that they
will. Moreover, we certainly expect that the implementation of both
the locking and the semaphore operations will need to change in an MP
environment, even if the interface remains the same.

It should be noted that the $x$-kernel, as currently implemented, is
MP-safe, although probably not MP-performant. This is because all
threads executing in the \xk{} must first acquire a master lock; i.e.,
the \xk{} is currently implemented as a single monitor. Our experience
suggests, however, that this implementation is reasonably efficient
for a small number of processors.

