% 
% trace.tex
%
% $Revision: 1.8 $
% $Date: 1993/02/03 22:46:44 $
%

\section{Trace Library}\label{kdebug}

The trace library provides a facility for tracing system events.  Most
importantly, the library allows conditional printing in {\em printf}
format of statements taking from zero to six variables.

\subsection{Type Definitions}

The current value of the tracing variable {\em tracevar} is used to
control whether or not a particular trace operation takes place.  The
trace variable values can be set at system build time; see section
\ref{config}. The following defined constants are suggestive of how to 
use trace levels:

\begin{tabbing}
xxxx \= xxxxxxxxxxxxxxxxxxxxxxxxxxx \= \kill
\>TR\_NEVER \>          for debugging statements that are unused (noop)\\
\>TR\_FULL\_TRACE\>     every subroutine entry and exit\\
\>TR\_DETAILED\>        all functions plus dumps of data structures at strategic points\\
\>TR\_FUNCTIONAL\_TRACE\>all the functions of the module and their parameters\\
\>TR\_MORE\_EVENTS\>    even more detail on events\\
\>TR\_EVENTS\>          more detail than major events\\
\>TR\_SOFT\_ERROR\>     mild warnings\\
\>TR\_MAJOR\_EVENTS\>   open, close, etc.\\
\>TR\_GROSS\_EVENTS\>   the coarsest tracing level\\
\>TR\_ERRORS\>          serious non-fatal errors; some residual event traces\\
\>TR\_ALWAYS\>          normally only used during protocol development
\end{tabbing}

\subsection{Operations}

\subsubsection{xTrace}

The {\em xTrace{\tt n}} macro takes {\tt n} arguments in addition to
the variables {\em tracevar}, {\em tracelevel}, and {\em
formatstring}.  {\em tracevar} is a name associated with the variable
being traced.  {\em tracelevel} is compared to the value of the trace
variable to determine at runtime if the trace statement should be
printed. {\em formatstring} is a {\em printf}-style formatting statement.

Each protocol has a trace variable based on the protocol name with
``p'' appended; e.g. udp has {\em traceudpp}.  In addition to protocol
tracing, there are $x$-kernel trace variables for subsystems: init,
processswitch, protocol, processcreation, cksum, event, system, msg,
ptbl, and sessn.  Most of these are defined in file {\sanss
xkernel/include/xk\_debug.h}.

Note that the trace facility automatically supplies a newline at the
end of the trace message; the supplied format string need not.  Also,
note that there should be no whitespace preceding a trace variable
name in any tracing statement.  Whitespace will cause errors in the
macro expansion and result in compilation errors.
\medskip

{\bold xTrace}{\em n}({\sem int} {\caps tracevar}, {\sem int} {\caps
tracelevel}, {\sem char} *{\caps formatstring}, {\caps args, ...} )
\medskip

\noindent For example:

\begin{verbatim}
	int traceudpp;

   xTrace2(udpp, TR_ERRORS, "input port %d output port %d", inp, outp);
\end{verbatim}

\subsubsection{xIfTrace}

If the {\em tracelevel} is less than or equal to the value of the {\em
tracevar}, then execute the statement directly following.
\medskip

{\bold xIfTrace} ({\sem int} {\caps tracevar}, {\sem int} {\caps tracelevel})

\medskip
\noindent For example:

\begin{verbatim}
        int traceudpp;

        xIfTrace(udpp, TR_ERRORS) 
                dump_header();
\end{verbatim}

\subsubsection{xAssert}

If the expression {\em exp} evaluates to FALSE, the $x$-kernel will
print a message and halt.
\medskip

{\bold xAssert} ({\sem bool} {\caps exp})

\subsubsection{xError}

Non-fatal error conditions can print warnings even in nondebugging
mode by using the {\em xError} call.
\medskip

{\bold xError} ({\sem char *} {\caps ErrorString})

\subsection{Usage Rules}

Trace and assert statements are macros which are only active in DEBUG
mode (see section \ref{config}).  In OPTIMIZE mode, debugging and
trace statements go away completely.  One should keep this in mind to
avoid bugs that show up only in OPTIMIZE mode.  For example, the
statment:

\begin{verbatim}
        xAssert( mapResolve(map, key, &p) == XK_SUCCESS );
\end{verbatim}

\noindent
will have no effect in OPTIMIZE mode.  One should be careful to
separate the operation and the check of the return code:

\begin{verbatim}
        res = mapResolve(map, key, &p);
        xAssert( res == XK_SUCCESS );
\end{verbatim}


If you are writing a new protocol, you should insert trace
statements. Even though there will never be bugs left after you
release your protocol, it may help others in debugging their protocol
which sits on top of, or below, yours. Don't delete these very helpful
debugging libraries when you are done.
