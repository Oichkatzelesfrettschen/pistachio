%
% ssr.tex
%
% $Revision: 1.3 $
% $Date: 1993/02/05 20:42:17 $
%

\subsection*{NAME}

\noindent SSR (a simple name server protocol for use with MachNetIPC).

\subsection*{SPECIFICATION}

See the file ``ssr.h''.  This contains the request identifier list
and the data structure defining an SSR message.  This is not MiG generated.

\subsection*{SYNOPSIS}

\noindent SSR is used for bootstrapping ports into MachNetIPC.  SSR
uses the Mach netnameserver to register its main port.  It listens
for messages on this port.  It can register a server port and forward
requests to its peers on remote hosts.

There is a simple name server that can be
reached using the Mach3 netnameserver, using the name ``simple\_server''.
This accepts Mach messages containing a service identifier, a request
identifier, an IP address, and a Mach reply port.  The requests are sent 
to the remote peer, and the Mach reply port is translated into a MachNetIPC
port.  The request is forwarded to the server registered on the remote
host (if any).  There is no reply to SSR.  

A test program using SSR can be found in the mach3/user/netipc/test 
directory.  This program ({\em ssr-test}) takes command line arguments
specifying the test type, the remote server address, the debugging
level, and the number of messages.

\subsection*{REALM}

SSR is in the anchor protocol realm; it is specific to the Mach3 platform.

\subsection*{CONFIGURATION}

SSR communicates with MachNetIPC as its only lower protocol.

\noindent {\tt name=ssr protocols=machripc;}

\subsection*{AUTHORS}

\noindent Hilarie Orman
