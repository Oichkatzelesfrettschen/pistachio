% 
% $RCSfile: api.tex,v $
%
% $Revision: 1.2 $
% $Date: 1993/02/05 22:55:22 $
%

\section{Application Programmer Interfaces}
\label{api}

This appendix describes the installation and use of \xk{} application
programmer interfaces.  API's in the \xk{} are platform-specific.


\subsection{ Mach 3 }

\subsubsection{Sockets}

The Mach 3 out-of-kernel platform provides a user socket library.
This library can be used to communicate with the XSI protocol
running in an out-of-kernel \xk{} in a separate task.
The combination of the library and the XSI protocol 
provides Berkeley socket semantics.  The mach 3 user-level build
will create a socket library in the {\tt xkernel/mach3/user/socket}
directory which can be linked with any application making Berkeley
socket calls.  These calls will then be routed through the \xk{}
socket implementation rather than the sockets in the BSD server.

The user-level build will also create some test programs in 
{\tt xkernel/mach3/user/socket/tst} that exercise the socket library
and the XSI protocol.

\subsubsection{Mach NetIPC}

The user-level build will create a test program, SSR-test, 
which exercises the \xk{}'s
out-of-kernel implementation of Mach netIPC in
{\tt xkernel/mach3/user/netipc/test}.

SSR-test communicates with the \xk{} protocols MachNetIPC and SSR.  
There must be an \xk{}
task with MachNetIPC running on the host, and there must be a 
Mach nameserver running on the host.

SSR-test takes command line arguments.  The first argument specifies
the test type.  The simplest test is the {\em postcard} test.  Other
test types are documented in the source code.

To start a postcard server at debugging level 1:

\noindent {\tt ssr-test -t p -l 1 \& };

To start a postcard client at debugging level 1, communicating with a
server at IP address 192.12.69.211, using 10 round-trips:

\noindent {\tt ssr-test -t p -p 192.12.69.211 -l 1 -trips 10 \& };


\subsubsection{Building}

The Mach 3 APIs can be built from a normal out-of-kernel \xk{} build
directory.  Running 

\begin{verbatim}
      make userdepend
\end{verbatim}

\noindent
will build the dependency files for the APIs and 

\begin{verbatim}
      make libUser
\end{verbatim}

\noindent
will build the socket library and the test programs.

