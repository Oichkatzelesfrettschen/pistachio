% 
% map.tex
%
% $Revision: 1.17 $
% $Date: 1993/02/05 22:28:35 $
%

\section{Map Library}

The map library provides a facility for maintaining a set of bindings
between identifiers.  The map library supports operations for
adding new bindings to the set, removing bindings from the set, and
mapping one identifier into another, relative to a set of bindings (lookup).
Protocol implementations use these operations to translate identifiers
extracted from message headers (e.g., addresses, port numbers) into
capabilities for kernel objects (e.g., {\em XObj}, {\em Enable}).

\subsection{Type Definitions}

The map library defines two data structures: {\em MapElement}
and {\em Map}. {\em Bind} is a pointer to a {\em MapElement}.
\medskip

\begin{tabbing}
xxxx \= xxxxxxxx \= xxxxxxxxxxxxx \= xxxxxxxxx \= \kill
\>{\sem typedef struct} mapelement \{\\
\>\>  {\sem struct mapelement} \>\>*next;\\
\>\>  {\sem void *} \>internalid;\\
\>\>  {\sem void *} \>externalid;\\
\>\} {\bold MapElement}, *{\bold Bind};\\
\\
\>{\sem typedef struct} \{\\
\>\>  {\sem int} \>nEntries, keySize;\\
\>\>  {\sem MapElement}  \>*cache;\\
\>\>  {\sem MapElement}  \>*freelist;\\
\>\>  {\sem MapElement}  \>**table;\\
\>\>  {\sem xkern\_return\_t} \>(*resolve)();\\
\>\>  {\sem Bind} \>(*bind)();\\
\>\>  {\sem xkern\_return\_t} \>(*unbind)();\\
\>\>  {\sem xkern\_return\_t} \>(*remove)();\\
\>\} *{\bold Map};
\end{tabbing}

\subsection{Operations}

\subsubsection{mapCreate}

Creates a map with {\em nEntries} elements in it. External ids bound
in this map are {\em keySize} bytes long. The maximum value for the
key size is MAX\_MAP\_KEY\_SIZE, currently 100 bytes.  Programmers will
normally use {\sf sizeof(structuretype)} as the keysize to facilitate
platform independence.  Note that maps never overflow, but they
perform best if {\em nEntries} is chosen so that the map is at most
50-80\% full.
\medskip

{\sem Map } {\bold mapCreate} ({\sem int} {\caps nEntries}, {\sem int} {\caps keySize});

\subsubsection{mapResolve}

Looks for the internal identifier bound to the external identifier
{\em ext} in the map {\em Map}.  Note that the resolution will be done
with {\em keysize} bytes of what the {\em ext} identifier points to,
where {\em keysize} is the parameter that was used in the {\em
mapCreate} call.  If a binding is found, {\em *ptr} is assigned the
value of the internal identifier and XK\_SUCCESS is returned.  If no
appropriate binding is found, mapResolve returns XK\_FAILURE.  If {\em
ptr} is NULL, only the error code is returned.
\medskip

{\sem xkern\_return\_t} {\bold mapResolve} ({\sem Map} {\caps Map}, 
{\sem void} *{\caps ext}, 
{\sem void} **{\caps ptr} );

\subsubsection{mapBind}

Adds a binding of external id {\em ext} to internal id {\em intern}
to map {\em Map}.  Note that the binding will be done with {\em
keysize} bytes of what the {\em ext} identifier points to, where {\em
keysize} is the parameter that was used in the {\em mapCreate} call.
The return value uniquely identifies this binding; it can later be
given as an argument to {\em mapRemoveBinding}.  
A return
value of ERR\_BIND indicates that the external identifier is already
bound in the map.
\medskip

{\sem Bind} {\bold mapBind} 
(
{\sem Map} {\caps Map}, 
{\sem void} *{\caps ext}, 
{\sem void *} {\caps intern}
);

\subsubsection{mapRemoveBinding}

Removes binding {\em Bind} from map {\em Map}.  Returns {\em
XK\_FAILURE} if the item is not in the map.
\medskip

{\em 3.1 Compatibility note}:  Same as 3.1 {\em mapunbindbinding}.
\medskip

{\sem xkern\_return\_t} {\bold mapRemoveBinding} ({\sem Map} {\caps Map}, {\sem Bind} {\caps Bind});

\subsubsection{mapUnbind}

Removes binding of the association key {\em ext} from the map {\em
Map}.  This is the inverse of {\em mapBind}.  
Returns XK\_FAILURE if the item is not in the map.
\medskip

{\em 3.1 Compatibility note:}  Same as 3.1 mapremove.
\medskip

{\sem xkern\_return\_t} {\bold mapUnbind} ({\sem Map} {\caps Map},  {\sem void } *{\caps ext});

\subsubsection{mapClose}

Destroys the map and frees its space.  Any elements left in the
map will be unbound before the map is destroyed.
\medskip

{\sem void} {\bold mapClose} ({\sem Map} {\caps Map});

\subsubsection{mapForEach}

Allows iterative access to the entries of a map by the provided
function {\em f}.  Each call to {\em mapForEach} puts the key and its
value into arguments passed the function {\em f}.  The third argument
passed to {\em f} is the supplied value {\em arg}.  As long as the
flag {\em MFE\_CONTINUE} is set in the function's return value
and there are unprocessed keys, {\em mapForEach}
will continue to call {\em f}.

If the flag {\em MFE\_REMOVE} is set in the return value, {\em mapForEach}
will remove the entry from the map after the user function returns and
before it is called with the next map entry.  This is the only
recommended way to remove the ``current'' map entry during a
{\em mapForEach} operation.  If the user function tries to remove the
``current'' entry directly (via mapRemove or mapUnbind), {\em mapForEach}
may fail to find all of the map entries.

New map entries added in the middle of a {\em mapForEach} iteration may or
may not show up during that iteration.  Map manipulations within a
{\em mapForEach} user function are generally not recommended.

{\em MFE\_REMOVE} and {\em MFE\_CONTINUE} are binary flags which may
be combined using bitwise OR.  A return value of 0 negates both flags.

The order in which keys are returned depends on the
internal structure of the map.

\medskip

{\sem void} {\bold mapForEach}({\sem Map} {\caps m}, {\sem MapForEachFun} {\caps f}, {\sem void} *{\caps arg} )
\medskip

{\sem typedef int} {\bold MapForEachFun}
(
{\sem void} *{\caps key}, 
{\sem void} *{\caps value}, 
{\sem void} *{\caps arg} 
)


\subsection{Usage Rules}

\subsubsection{External Keys}

Maps are used to bind a variable length external id to an internal id
of type {\em int}.  The size of the external id is given
as an argument when a particular map is created. All external ids
bound using this map are expected to be of this size.  The user is
warned to use a zero-izing routine like {\sanss bzero} before
assigning values to a structure that will be used with the map
routines.  The C language can have uninitialized data in the
interstices of structures (i.e., padding areas), and these can cause
structures that are ``equal'' (i.e., all fields have the same values)
to fail to map to the same value in the $x$-kernel.


\subsubsection{Active and Passive Maps}

Protocols generally maintain two maps: an active map and a passive
map.  Active maps are used to map keys found in incoming messages into
the session that will process the message. Thus, the active map holds
information about the set of currently active connections. Passive
maps are used to bind identifiers to {\em Enable} objects 
(section \ref{enable_objects}),
thereby allowing a
protocol to create a session when a message that is part of a new
connection arrives.

Typically, a protocol binds an active key to a session in its {\em
open} routine, and a passive key to an enable object in its {\em
openEnable} routine. These bindings are then used in the protocol's
{\em demux} operation, according to the general pattern illustrated in
figure~\ref{fig:mapfig}.

\begin{figure}
\caption{Example of using maps in demux \label{fig:mapfig}}
\begin{verbatim}
static xkern_return_t
yap_demux(self, lls, msg)
    XObj self;
    XObj lls;
    Msg *msg;
{
    Enable  *e;
    
    /* get protocol-specific state from self */
    pstate = self->state;

    /* extract the header from the message */
    msgPop(msg, yapHdrLoad, &hdr, HLEN, 0);

    /* make an active key from fields in the header (and other info) */
    /* then see if it is in active map; if found, call pop */
    activeid.localport = hdr.dstport;
    activeid.remoteport = hdr.srcport;
    activeid.lls = lls;
    if (mapResolve(pstate->activemap, &activeid, &s) != XK_FAILURE)
    {  
        xkr = xPop(s, msg, lls, &hdr);
        return xkr;
    }

    /* make a passive key from fields in the header (and other info) */
    /* then for it in the passive map */
    /*  if found, create a new session, */
    /* save a binding to it in the active map, */
    /* and call opendone and pop */
    passiveid = hdr.dstport;
    if (mapResolve(pstate->passivemap, &passiveid, &e) != XK_FAILURE) 
    {
        s = yapCreateSessn(self, e->hlpRcv, e->hlpType, &activeid);
        xDuplicate(lls);
        s->binding = mapBind(pstate->activemap, &activeid, (int)s);
        xOpenDone(e->hlpRcv, s, self);
        xkr = xPop(s, lls, msg, &hdr);
        return xkr;
   }

    /* don't know how to process this message, so drop it */
    return;
}
\end{verbatim}
\end{figure}

Notice two things about the example. First, in the
case where a new session is created, it is necessary to call {\em
xDuplicate} on the lower-level session to make the reference we hold
to it permanent (see Sections \ref{duplicate} and \ref{refcnt}). 
Second, it is useful
to save the binding of an active key to the newly created session in
the session object itself. 
(The same thing is done when a session is created in the
protocol's {\em open} routine).
This is because it will be necessary to be able to remove the binding from the
active map when the session is closed, and this is often easier or
more efficient if the
binding itself is available.
Finally, the active key in this
example includes both fields found in the message header and the 
lower-level session that called xDemux. Using the lower-level session as
part of the active key is fairly common; it helps to disambiguate
protocol header fields that are not unique across all lower-level
connections.
