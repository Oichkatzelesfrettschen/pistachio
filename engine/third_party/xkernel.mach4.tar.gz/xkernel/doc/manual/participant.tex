% 
% $RCSfile: participant.tex,v $
%
% x-kernel v3.2
%
% Copyright (c) 1993,1991,1990  Arizona Board of Regents
%
%
% $Log: participant.tex,v $
% Revision 1.9  1993/11/30  17:20:36  menze
% Added comment about participant conventions in passive opens
%


\section{Participant Library}\label{part}

Participant lists identify members of a session and are used for
opening connections.  Upper protocols interested in establishing a
connection will construct the participant list and pass it to the
lower protocol as a parameter of an open routine.  The lower protocol
will then extract information from the participant list, possibly
passing the participant list on to its own lower protocol.

Each participant in the list contains a participant address stack,
designed to facilitate a general method of communicating encapsulated
address information between protocol layers.  By using pointers to
address information, one layer can pass address information through a
lower layer without having the lower layer manipulate the address
information at all, not even by copying.  The address information for
each participant is kept as a stack of {\em void *} pointers to
address components and the lengths of each component.
The component pointers are pushed or popped onto
the stack by utility functions.

\subsection{Type Definitions}

The participant data structure is used to collect addressing
information for opening connections.  A participant list is defined to
be an array of type {\em Part}, and a {\em PartStack} is the main
field in a single {\em Part}. The fields of these structures should
not be directly accessed by the protocol developer.

\begin{tabbing}
xxxx \= xxxx \= xxx \= xxxxxx \= xxxxxxxxxxxxxxxxxxxx \= \kill
\>\#define\>\>\>PART\_MAX\_STACK\>20\\

\>{\sem typedef struct}  \{\\
\>\>    struct \{\\
\>\>       \>{\sem void *}     \>ptr;\\
\>\>       \>{\sem int}        \>len;\\
\>\>    \} {\caps arr}[PART\_MAX\_STACK];\\
\>\>    {\sem int}	\>\>{\caps top};\\
\>\} {\bold PartStack};\\
\\
\>{\sem typedef struct} \{\\
\>\>    {\sem int}	\>\>{\caps len};\\
\>\>    {\sem PartStack}\>\>{\caps stack}; \\
\>\} {\bold Part};
\end{tabbing}

\subsection{Participant List Operations}

The following operations provide a convenient interface that hides the
{\em PartStack} data structure. However, the fact that a participant
list is really an array of type {\em Part} is visible to the
programmer.

\subsubsection{partInit}
Initialize participant list {\em p} of {\em N} participants.\medskip

{\sem void} {\bold partInit}({\sem  Part *}{\caps p}, {\sem int} {\caps N} );
\medskip

\subsubsection{partPush}
Pushes address {\em addr}, pointing to {\em len} bytes, onto the 
stack of participant {\em p}.   A length of 0 indicates a ``special-value''
pointer (e.g., ANY\_HOST) whose value as a
pointer should be preserved across protection boundaries (and not
dereferenced.)  See section \ref{part_usage} below.

\medskip

{\sem void} {\bold partPush}
( {\sem Part } {\caps p}, {\sem void *} {\caps addr}, {\sem int} {\caps len})
\medskip

\subsubsection{partPop}
Pops an address off the stack of participant {\em p}.  Returns NULL if
there are no more elements on the stack.
\medskip

{\sem void *} {\bold partPop}({\sem Part }{\caps p} )
\medskip


\subsubsection{partStackTopByteLen}
Returns the number of bytes pointed to by the top element of the stack
of participant {\em p}.  Returns 0 if the stack element was pushed
with a length field of zero (i.e., a ``special-value'' pointer.)
Returns -1 if there are no elements on the stack.
\medskip

{\sem int} {\bold partStackTopByteLen}({\sem Part }{\caps p} )
\medskip


\subsubsection{partLen}
Reports the number of participants in participant list {\em p}.\medskip

{\sem int} {\bold partLen}( {\sem Part *} {\caps p} )
\medskip


\subsubsection{partExternalize}
Takes a participant list {\em p} and externalizes it
into {\em buf} in a form appropriate for transmission across a
protection boundary, essentially dereferencing all the pointers
except special-value pointers pushed with length == 0.  The {\em int}
 pointed to by {\em len} holds
the size of the buffer going in and holds the number of bytes written
to the buffer on return.  If the buffer is not large enough, 
{\em partExternalize} will fail.
\medskip

{\sem xkern\_return\_t} {\bold partExternalize}
( {\sem Part *} {\caps pList}, {\sem void *} {\caps buf}, 
  {\sem int *} {\caps len} )
\medskip


\subsubsection{partInternalize}
Takes an externalized participant buffer {\em buf} resulting
from {\em partExternalize} and recreates the participants in the
uninitialized participant list {\em p}.
\medskip

{\sem void} {\bold partInternalize}
( {\sem Part *} {\caps p}, {\sem void *} {\caps buf} )
\medskip


\subsubsection{partExtLen}
Reports the number of participants in an externalized participant
buffer {\em buf} resulting from {\em partExternalize}.  
This can be useful for allocating the appropriate
sized participant list to pass to {\em partInternalize}.
\medskip

{\sem int} {\bold partExtLen}( {\sem void *} {\caps buf} )
\medskip


\subsection{Relative Protocol Numbers}
\label{relprotnum}

Participant lists are used for passing addressing information between
protocols. An additional problem is how a high-level protocol
identifies {\em itself} to a low-level protocol.  In most conventional
protocols, a low-level protocol uses a {\em relative protocol number}
to identify the protocols above it; e.g., IP identifies UDP with
protocol number 17 and TCP as protocol number 6. However, protocols
that have been especially designed to use the \xk{} use an
{\em absolute} addressing scheme.

Version 3.2 of the \xk{} reconciles these two approaches by
maintaining a table of relative protocol numbers.  (See Section
\ref{protnum} for the format of this table.)  Rather than embed
protocol numbers in the protocol source code, protocols learn the
protocol numbers of protocols above them by querying this table using
the following operation:
\medskip

{\sem long}	{\bold relProtNum}( {\sem XObj} {\caps hlp}, {\sem XObj} {\caps llp} )
\medskip

\noindent This operation returns the protocol number of the high-level
protocol relative to the low-level protocol, or -1 if no such binding
has been configured in the protocol tables.  This number will have to
be cast into the appropriate type; e.g., an unsigned short by the
ETH protocol and an unsigned char by IP.

\subsection{Usage Rules}
\label{part_usage}

By convention, active participant lists (those used in {\em xOpen})
have the remote participant(s)
first followed by an optional local participant.  The local
participant can often be omitted, in which case the protocol tries to
use a reasonable default.  For example, a UDP participant contains a
UDP port and an IP host.  If the local participant is missing from an
active participant list, UDP selects an available port for the local
participant.

In some cases, it is necessary to specify part of the information in a
participant, but it is convenient to allow the lower protocol to
``fill in'' the rest.  To allow this flexibility, one can use the
constant pointer ANY\_HOST to specify ``wildcard'' hosts.  For
example, if one wants to open UDP with a specific local port, but does
not care which local host number is used, one could construct a local
participant with the specific local port but with the pointer
ANY\_HOST pushed on the stack.  The protocol that interprets the host
part of the participant stack could then choose a reasonable default.
Similarly, a wildcard value ANY\_PORT exists for protocols that use ports
on their stacks.  Protocols that support wildcards indicate this
in their manual page.

The following example illustrates how a protocol that is about to invoke
{\em xOpen} on a low-level protocol initializes the participant list, and 
then how the low-level protocol extracts that information from the
participant list.

\begin{verbatim}
{
    ...
    /* set participant addresses before calling low-level protocol's open */
    partInit(p, 2);
    partPush(p[0], &ServerHostAddr, sizeof(IPhost));  /* remote */
    partPush(p[0], &ServerPort, sizeof(long));        /* remote */
    partPush(p[1], ANY_HOST, 0);                      /* local  */
    partPush(p[1], &ClientPort, sizeof(long));        /* local  */
    xOpen(self, self, llp, p);
    ...
}


llpOpen( XObj self, XObj hlpRcv, XObj hlpType, Part *p )
{
    /* get participant addresses within low-level protocol's open */
    remoteport = (long *) partPop(p[0]);
    localport = (long *) partPop(p[1]);
    ...
}
\end{verbatim}

Notice in this example how the high-level protocol pushes two 
items (a host address and a port number) onto each participant's
address stack, but the low-level protocol pops only one item off.
This is because the low-level protocol does not interpret the first
item (the host address); 
it just passes it on to {\em its} low-level protocol. Also note
that when using participants that have been passed from other
protocols, the user must keep in mind that the address pointers may be
valid only for the duration of the current subroutine.  Data that is
needed beyond this time should be explicitly copied into static
storage.  In addition, because the participant structure is passed by
reference in the {\em xOpen} call, the caller should consider the
contents invalid after the return.

In general, passive participant lists (those used in {\em
xOpenEnable}) contain only the local participant, with no remote
participant specified.  This indicates that an upper protocol is
willing to accept connections from any remote participant, as long as
the connection is addressed to the correct local participant.
Protocols which provide different semantics for their openEnable
participants will indicate this explicitly in their manual page in
appendix \ref{protman}.

