%
% debug.tex
%
% $Revision: 1.4 $
% $Date: 1992/02/12 23:48:50 $
%

\section{Debugging a Kernel}

\subsection{Tracing}
\label{debug}
The $x$-kernel uses a trace package to generate debugging information.
To enable the tracing facility, edit the Makefile to set {\tt
HOWTOCOMPILE} to {\tt DEBUG}; tracing can be disabled by setting {\tt
HOWTOCOMPILE} to {\tt OPTIMIZE}. Then type {\tt make}.  See also
section \ref{kdebug}.

If you are interested in accurate
performance timings, you should set {\tt HOWTOCOMPILE} to {\tt
OPTIMIZE} in the Makefile.  This causes all trace of tracing code to
be eliminated form the kernel.

If you are writing a new protocol, you should insert trace statements.
(Even though there will never be bugs left after you release your
protocol, it may help other in debugging their protocol which sits
on top of, or below, yours.)  Don't delete these very helpful debugging
libraries when you are done.

\subsection{Memory Leaks}

The memory debugging library has not yet been ported to either the
sunos or the mach3 platform.
