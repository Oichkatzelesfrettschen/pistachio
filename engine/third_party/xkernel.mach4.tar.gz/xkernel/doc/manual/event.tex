% 
% event.tex
%
% $Revision: 1.14 $
% $Date: 1993/11/18 06:05:20 $
%

\section{Event Library}

The event library provides a mechanism for scheduling a procedure to
be called after a certain amount of time. By registering a procedure
with the event library, protocols are able to ``timeout'' and act on
messages that have not been acknowledged or to perform periodic
maintenance functions. 

\subsection{Type Definitions}

The only event-related type that protocol programmers need be
aware of is the type {\em Event}. However, this type is defined
by the underlying platform and is opaque to the protocol programmer.

\subsection{Operations}

\subsubsection{evSchedule}

Schedules an event that executes function {\em f} with argument {\em a}
after delay {\em t} usec; {\em t} may equal 0.  A handle to the event
is returned, and this can be used to cancel the event at some later
time.  When an event fires, a new thread is created to run function
{\em f}. Note that even after an event fires and a thread had been
scheduled to handle it, the thread does not run until sometime
after the currently executing thread gives up the processor.  See
Section \ref{thread} for a description of how threads are scheduled.
\medskip

{\sem Event} {\bold evSchedule} ({\sem EvFunc} {\caps f}, {\sem void}*
{\caps a}, {\sem int} {\caps t})

{\sem typedef void} ({\bold * EvFunc} )
( {\sem Event} *, {\sem void} * )


\medskip

Function {\em f} must be of type {\em void} and take two arguments:
the first, of type {\em Event}, is a handle to the event itself
and the second,
of type {\em void *}, is the argument passed to evSchedule.  
In order to satisfy the C compiler type checking
rules when accessing the arguments, function {\em f} must
begin by casting its second argument to be a non-void type.

\subsubsection{evDetach}

Releases a handle to an event. As soon as {\em f} completes, the
internal resources associated with the event are freed.  All events
should eventually be either detached or canceled
to assure that system resources are released.
\medskip

{\sem void}  {\bold evDetach} ({\sem Event} {\caps e})
\medskip

\subsubsection{evCancel}

Cancels event {\em e} and returns EVENT\_FINISHED if the event has
already happened, EVENT\_RUNNING if the event is currently running, and
EVENT\_CANCELLED if the event has not run and can be guaranteed to not
run. In the case where {\bold evCancel} returns EVENT\_RUNNING, the
caller must be careful to not delete resources required by the event.
\medskip

{\sem EvCancelReturn} {\bold evCancel} ({\sem Event} {\caps e})


\subsubsection{evIsCancelled}

Returns true if an {\bold evCancel} has been performed on the event. 
Because event handlers receive their event as the first calling
argument, it is possible for a handler to check for cancellation
of itself from other threads.

\medskip

{\sem int} {\bold evIsCancelled} ({\sem Event} {\caps e})


\subsubsection{evDump}

Displays a {\em ps}-style listing of \xk{} threads in \xk{}'s compiled
with DEBUG mode.  The address of the entry function, the thread state
(pending, scheduled, running, finished, or blocked), the
time relevant to the thread state, and flags (detached or cancelled), are
displayed for each thread controlled by \xk{} monitor.  The meaning
of the time entry varies according to the state.  For pending threads,
the time is the time until it will be scheduled; for other states
it is the time the thread has spent in that state.  The time is reset
on each transition, i.e., it is not cumulative.

\medskip

{\sem void} {\bold evDump} ({\sem void})



\subsection{Usage Rules}

\subsubsection{Repeating Events}

Each event that is scheduled executes at most one time. Repeating
events are programmed as follows:

\begin{verbatim}

        foo_init()
        {
            ...
            evDetach( evSchedule( f, argument, INTERVAL ) );
        }
        
        
        f( Event self, void *arg )
        {
            actual work
            ...
            evDetach( evSchedule( f, arg, INTERVAL ) );
        }
        
\end{verbatim}


\subsubsection{Cancellable Events}

The evIsCancelled routine is designed to make it easy to write events
which might be cancelled before (or while) they run.  It is common
practice, for example, for a session to pass session state to a
timeout event.  The evIsCancelled notification can be used to
synchronize the timeout event and the possible destruction of the
session state.  Here is an example:


\begin{verbatim}

        foo_destroy()
        {
            ...
            evCancel( state->timeoutEvent );
            ...
        }
        
        
        foo_timeout( Event self, void *state )
        {
           ...
           xPush(lls, retransmitMsg);
           /* 
            * xPush may have blocked -- check to see if state is 
            * still valid
            */
            if ( evIsCancelled(self) ) {
                return;
            }
            state->timeoutEvent = evSchedule( foo_timeout, arg, INTERVAL );
        }
        
\end{verbatim}


\subsubsection{Event Granularity}

Although the event library uses an efficient representation (timing
wheels) protocol programmers should be careful to not schedule events
that are too fine grained. For example, in TCP, it is better to schedule
one event for every session rather than for every message that is
sent.
