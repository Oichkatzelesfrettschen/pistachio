% 
% $RCSfile: api.tex,v $
%
% $Log: api.tex,v $
% Revision 1.11  1993/11/29  21:25:55  hkaram
% Minor spelling fix
%
% Revision 1.10  1993/11/23  23:39:37  hkaram
% Changes to nnstest docs.
%
% Revision 1.9  1993/11/23  23:04:34  menze
% Spelling corrections
%
%

\section{Application Programmer Interfaces}
\label{api}

This appendix describes the installation and use of \xk{} application
programmer interfaces.  APIs in the \xk{} are platform-specific.


\subsection{ Mach 3 }

\subsubsection{Sockets}

The Mach 3 out-of-kernel platform provides a user socket library.
At this time, the library can only be compiled with a DECstation 5000. 
The library can be used to communicate with the XSI protocol
running in an out-of-kernel \xk{} in a separate task.
The combination of the library and the XSI protocol 
provides Berkeley socket semantics.  The mach 3 user-level build
will create a socket library in the {\tt xkernel/mach3/user/socket}
directory which can be linked with any application making Berkeley
socket calls.  These calls will then be routed through the \xk{}
socket implementation rather than the sockets in the BSD server.

The user-level build (see below) will also create some test programs 
that exercise the socket library
and the XSI protocol in {\tt xkernel/mach3/user/socket/tst}.  
Two versions of each test program are created,
one linked against the \xk{} socket library and one linked against the
standard Mach libraries.  An \xk{} configured with the XSI protocol
must be running for the \xk{} versions of the test programs to run, or
a message of the form:
\begin{quote}
\begin{tt}
xsi\_user\_init: netname\_lookup(): (ipc/mig) server died
\end{tt}
\end{quote}
will be displayed.  The test programs have a number of flags
controlling their behavior.  Invoking them with {\tt -h} will give a
list of these parameters.


\subsubsection{MachNetIPC}

The user-level build (see below) will create a test program, nnstest,
which exercises the \xk{}'s
out-of-kernel implementation of MachNetIPC in
{\tt xkernel/mach3/user/netipc/test}.

NNS test communicates with the \xk{} protocols NNS which is the top
layer of MachNetIPC.
There must be an \xk{} task with MachNetIPC running on the host.

NNS test takes command line arguments. Before performing checkins, checkouts,
and lookups the test program first looks up the service ``NetworkNameServer''
and uses the associated port as the default name server port. This is to
avoid conflicts with a previously running simple name server. Following are
some example runs.

\bigskip

\noindent
To checkin a service on the local host do:

\begin{quote}
{\tt nnstest -i service\_name \& };
\end{quote}

\noindent
Note that to perform remote host lookups, you have to specify the 
IP address of the host and not the host name. To look up a service 
on a remote machine with IP address 192.12.69.88 do:

\begin{quote}
{\tt nnstest -h192.12.69.88 -l service\_name };
\end{quote}

\noindent
To checkin, lookup and checkout a service on the remote machine do:

\begin{quote}
{\tt nnstest -h192.12.69.88 -ilo service\_name };
\end{quote}

\subsubsection{Building}

The Mach 3 APIs can be built from a normal out-of-kernel \xk{} build
directory.  Running 

\begin{verbatim}
      make userDepend
\end{verbatim}

\noindent
will build the dependency files for the APIs and 

\begin{verbatim}
      make user
\end{verbatim}

\noindent
will build the user libraries and the test programs.  Building the
APIs should only be attempted {\em after} a successful build of the
rest of the \xk{} system.

