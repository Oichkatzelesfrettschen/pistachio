% 
% $RCSfile: protNums.tex,v $
%
% x-kernel v3.2
%
% Copyright (c) 1993,1991,1990  Arizona Board of Regents
%
%
% $Log: protNums.tex,v $
% Revision 1.6  1993/11/30  17:24:35  menze
% Became a subsection
%
% Wordsmithing by Larry
%
% Described the two distributed protocol tables
%

\subsection{Protocol Tables}
\label{protnum}

The \xk{} runtime environment always includes a protocol table.
For each protocol, this table defines a numbering space for
identifying other protocols.  The \xk{} builds a table of protocol
numbers by reading configuration files at boot time, and provides an
operation for protocols to query this table (see section
\ref{relprotnum}).  This frees protocols from embedding explicit
protocol numbers in the protocol code itself.

If you have not configured any non-standard protocols in your graph,
you can safely use one of the default protocol tables in {\sanss
/usr/xkernel/etc} and may skip to section \ref{protTblRomOpt}.

The following is an example configuration
file defining a protocol table:

\begin{verbatim}
#
# prottbl
#
# This file describes absolute protocol id's 
# and gives relative protocol numbers for those
# protocols which use them
#
# x-kernel v3.2

eth         1    
{
        ip         x0800
        arp        x0806
        rarp       x8035
        #
        # ethernet types x3*** are not reserved
        #
        blast      x3001
}
ip          2
{
        icmp       1
        tcp        6
        udp        17
        #
        # IP protocol numbers n, 91 < n < 255, are unassigned
        #
        blast      101
}
arp         3
rarp        4
udp         5
tcp         6
icmp        7
blast       8
sunrpc      9

\end{verbatim}

\noindent Each protocol has an entry of the form:

\medskip

\begin{quote}
{\tt name   idNumber    [ \{  hlp1  relNum1  hlp2 relNum2  ... \} ] }
\end{quote}

\medskip

\noindent where the idNumber uniquely identifies each protocol.

There are two ways for a protocol to define its protocol
number space.  
The first technique uses an {\em explicit numbering} scheme; the
protocol explicitly indicates which protocols may be configured above
it and the relative number that should be used for each higher-level
protocol.  
Use of this numbering scheme
is indicated by the presence of the optional hlp list after the
protocol's idNumber.

The second technique uses an {\em implicit numbering} scheme; the
protocol does not explicitly name its allowed upper protocols, but
will implicitly use each protocol's unique idNumber as its relative protocol
number.  BLAST, for example, employs this technique and would use
SUNRPC's idNumber 9 as its protocol number relative to BLAST.
Protocol ID numbers are 4-byte quantities in the \xk{}, so protocols 
using implicit numbering have a 4-byte field in their headers for the
upper protocol number.

Implicit numbering clearly allows more flexibility in how protocols
may be composed.  As flexible composability is one of the goals of the
\xk{}, all new protocols written in the \xk{} should use this implicit
numbering scheme.

The \xk{} is distributed with two useful configuration files for
building protocol tables, located in {\sanss /usr/xkernel/etc}: 
{\sanss prottbl.std} and {\sanss
prottbl.nonstd}.  The first of these two files, {\sanss
prottbl.std} contains the standard protocol numbers; e.g., TCP is
known as protocol ``6'' relative to IP, IP is known as protocol ``x0800''
relative to the ethernet, and so on. This file should be used when
there is no danger of interfering with the standard protocol suite on
the platform you are using. The second file, {\sanss prottbl.nonstd},
uses non-standard protocol numbers. You will want to use this file
when there is a danger of interfering with the platform's native
protocol stack; e.g., both the \xk{} and the native stack are running
and getting packets diverted their way by a packet filter.


All \xk{}s must load at least one protocol table, and 
all protocols configured in a kernel must have an entry in the
protocol table.   
If you are
writing a new protocol, you will probably want to define an auxiliary
protocol table which assigns a temporary idNumber to your new protocol
and then configure your kernel to read in both the system table and
your auxiliary table (see section \ref{compose} for configuration
examples.)  If you need to configure your new protocol above an
existing protocol which uses explicit numbering, you can augment the
table for the existing protocol as in this example:

\begin{verbatim}
#
# Auxiliary protocol table 
#

yap	1000

ip	2 	{
        yap	200
}
\end{verbatim}

Here, we define our new protocol YAP and indicate that it has protocol
number 200 relative to IP.  Note that when augmenting the tables of
other protocols, the idNumber of the other protocol must match its
number in the system file.

The \xk{} runs consistency checks on the protocol tables and will give
error messages and abort in the presence of inconsistencies.



\subsubsection{ROM options}
\label{protTblRomOpt}

Protocol table configuration files can be specified in either
graph.comp (see section
\ref{config}) or in a ROM file (see section \ref{romfile}).  The ROM
file format is:

\begin{quote}
\begin{tt}
prottbl FILENAME1\\
prottbl FILENAME2\\
...
\end{tt}
\end{quote}

\noindent
In both cases, the named protocol tables are loaded at run-time.
Platforms for which loading protocol tables at run-time is impractical
(e.g., inside the Mach kernel) provide platform-specific mechanisms
to support protocol tables (see, for example, section
\ref{installingmachk}). 

