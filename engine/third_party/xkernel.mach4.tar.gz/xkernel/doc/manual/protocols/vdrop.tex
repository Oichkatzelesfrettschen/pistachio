%     
% $RCSfile: vdrop.tex,v $
%
% x-kernel v3.2
%
% Copyright (c) 1993,1991,1990  Arizona Board of Regents
%
%
% $Revision: 1.1 $
% $Date: 1993/10/29 17:07:46 $
%

\subsection*{NAME}

\noindent VDROP (Virtual Drop Protocol)

\subsection*{SPECIFICATION}

\noindent 
Throws away occasional incoming packets.  Used to exercise the
recovery mechanisms of other protocols.

\subsection*{SYNOPSIS}

\noindent 
VDROP sessions throw away incoming packets at regular intervals.  This
interval is set in a somewhat random fashion at session creation time,
though it can be set explicitly on a per-session basis via a control
operation. 

VDROP has no effect on outgoing packets.

VDROP should allow its interval to be set via a ROM option and should
allow sessions to have more interesting distributions of drop
intervals.

 
\subsection*{REALM}

VDROP is in the ASYNC realm.

\subsection*{PARTICIPANTS}

VDROP passes participants to the lower protocols without manipulating
them. 

\subsection*{CONTROL OPERATIONS}

\begin{description}

\item[{\tt VDROP\_SETINTERVAL:}]
Sets the drop interval for this session.  An interval of 1 drops every
packet, an interval of 2 drops every other packet, etc.  An interval
of zero indicates that VDROP is disabled for that session.  (session only)
\begin{description}
\item[{\rm Input:}] {\tt int interval}
\item[{\rm Output:}] none
\end{description}


\item[{\tt VDROP\_GETINTERVAL:}]
Returns the current drop interval for this session.
(session only)
\begin{description}
\item[{\rm Input:}] none
\item[{\rm Output:}] {\tt int interval}
\end{description}


\end{description}



\subsection*{AUTHOR}

\noindent Ed Menze
