%
% $RCSfile: blast.tex,v $
%
% $Revision: 1.11 $
% $Date: 1993/11/18 06:04:58 $
%

\subsection*{NAME}

\noindent BLAST (RPC Blast Micro-Protocol)

\subsection*{SPECIFICATION}

\noindent 
S. O'Malley and L. Peterson.  A Dynamic Network Architecture.  {\it
ACM Transactions on Computer Systems 10}, 2 (May 1992), 110--143.

\medskip
\noindent  B. Welch. {\it The Sprite remote procedure call}. 
University of California at Berkeley, Tech Report UCB/CSD 86/302,
June 1986.

\subsection*{SYNOPSIS}

\noindent 
BLAST is a micro-protocol version of Sprite RPC's
fragmentation algorithm. The algorithm was extracted from Sprite and
made into a stand-alone protocol. BLAST takes a large message,
fragments it into smaller packets, and sends them. 
The maximum packet size accepted by BLAST (as returned by the GETMAXPACKET
control op) is the product of the maximum number of fragments handled
by BLAST (16 by default) and the optimal packet size of BLAST's lower
protocol. Blast is tuned for the local area networks and should 
not be used across the Internet. 

The receiver gathers all of the packets and sends a NACK if it has
reason to believe (through time-outs or other considerations) that
a packet has been dropped.
BLAST can handle any
number on outstanding messages between two hosts (buffer space
permitting, of course). The protocol is bidirectional; i.e., it
supports blasts in both directions over the same session. Small
messages take a short cut through the protocol and do not require the
allocation of any resources.  

The sender keeps a copy of the message
around until a time-out occurs or the higher level protocol that sent
the message notifies BLAST that it can free the message (through a
FREERESOURCES control op.) Users of blast are strongly encouraged 
to free messages as soon as possible. The sender knows which 
BLAST (BLAST can be instantiated more than once) and which message
to free because when a push was performed blast writes into a message
attribute attached by CHAN (or some other high level protocol)  a pointer 
to itself and a 32 bit integer ticket which uniquely identifies the message. 

Because the sending BLAST may time-out and release a message before all
fragments have been received, BLAST is not reliable. 
It is, however, very persistent.

BLAST performance is critically dependant upon the time-out 
strategy used and the initial values of those timers. As mentioned 
earlier the sender uses a timer to free resources after a set 
interval has occurred. Tuning this timer for use with higher 
level protocols which do not explicitly free resources is very 
difficult. For applications which do free resources this time-out 
interval has no effect on performance unless it is set to too small a 
value. The receiver sets a timer whenever a fragment from a 
new packet arrives. The only purpose of this timer is to detect 
the drop of the last fragment. This timer is set to some constant
times the number of fragments in the message. If this timer 
expires to early this is detected by the code and the constant 
is increased by a factor of two. After a NACK is set to the 
round trip time plus some constant times the number of 
fragments. The purpose of this time is to generate a new NACK
if the original NACK or retransmitted segments are lost. 


\subsection*{REALM}

BLAST is in the ASYNC realm.

\subsection*{PARTICIPANTS}

BLAST neither removes nor adds anything to the participant stacks.

\subsection*{CONTROL OPERATIONS}

\begin{description}

\item[{\tt FREERESOURCES:}]
Free the storage associted with the message handle passed as 
argument. The handle should be a value returned by xPush. 
(protocol and session).
\begin{description}
\item[{\rm Input:}] {\tt xmsg\_handle\_t} 
\item[{\rm Output:}] none
\end{description}

\item[{\tt CHAN\_RETRANSMIT:}]
This is CHAN's way of asking BLAST if it should go ahead and retransmit
the message. BLAST returns true (1)  if and only it has received no 
NACK's for this message since the message was sent or the last time 
CHAN\_RETRANSMIT was called. The idea being that CHAN should not 
retransmit while BLAST is in the process of sending the message.
\begin{description}
\item[{\rm Input:}] none
\item[{\rm Output:}] {\tt 0 or 1} 
\end{description}

\item[{\tt BLAST\_SETOUTSTANDINGMSGS:}]
Set the number of outstanding messages allowed (protocol only).
\begin{description}
\item[{\rm Input:}] {\tt int} 
\item[{\rm Output:}] none
\end{description}

\item[{\tt BLAST\_GETOUTSTANDINGMSGS:}]
Get the number of outstanding messages allowed (protocol only).
\begin{description}
\item[{\rm Input:}]  none
\item[{\rm Output:}] {\tt int} 
\end{description}

\end{description}

\subsection*{CONFIGURATION}

BLAST requires only its lower transport protocol.  Since BLAST doesn't
use host addresses, it can sit on top of protocols using different
address types without modification.

\subsection*{AUTHORS}

\noindent Sean O'Malley and Ed Menze


