%
% udp.tex
%
% $Revision: 1.7 $
% $Date: 1993/09/18 03:23:07 $
%

\subsection*{NAME}

\noindent UDP (User Datagram Protocol)

\subsection*{SPECIFICATION}

\noindent J. Postel. {\it User Datagram Protocol}. Request for Comments 
768, USC Information Sciences Institute, Marina del Ray, Calif., Aug. 1980.

\subsection*{SYNOPSIS}

\noindent UDP is a trivial protocol that dispatches messages that arrive 
at the host to a process running on the host.

\subsection*{REALM}

UDP is in the ASYNC realm.

\subsection*{PARTICIPANTS}

UDP removes a pointer to a long (the UDP port number) from the
participant stack.  UDP ports must be less than 0x10000.  
If the local participant is missing, or if the
local protocol number is ANY\_PORT, UDP will select an unused local
port. 


\subsection*{CONTROL OPERATIONS}

\begin{description}

\item[{\tt UDP\_ENABLE\_CHECKSUM:}]
Cause the session to use checksums on its outgoing packets.  
(session only)
\begin{description}
\item[{\rm Input:}] none
\item[{\rm Output:}] none
\end{description}

\item[{\tt UDP\_DISABLE\_CHECKSUM:}]
Cause the session to not use checksums on its outgoing packets.
(session only) 
\begin{description}
\item[{\rm Input:}] none
\item[{\rm Output:}] none
\end{description}


\item[{\tt UDP\_GETFREEPROTNUM:}]
Returns an unused UDP port number.  This port number will not be given
out to subsequent UDP\_GETFREEPROTNUM calls until it is released with
UDP\_RELEASEPROTNUM.  This allows an opener to separate reservation of
free ports from the actual open operation, if desired.
(protocol only) 
\begin{description}
\item[{\rm Input:}] none
\item[{\rm Output:}] long
\end{description}


\item[{\tt UDP\_RELEASEPROTNUM:}]
Releases a UDP portnumber previously acquired with UDP\_GETFREEPROTNUM.
(protocol only)
\begin{description}
\item[{\rm Input:}] long
\item[{\rm Output:}] none
\end{description}



\end{description}

\subsection*{CONFIGURATION}

\noindent {\tt name=udp protocols=ip;}

\subsection*{AUTHORS}

\noindent Larry Peterson and Sean O'Malley
