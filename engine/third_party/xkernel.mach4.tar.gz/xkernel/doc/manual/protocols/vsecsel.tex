% $RCSfile: vsecsel.tex,v $
%
% $Revision: 1.3 $
% $Date: 1993/11/08 23:58:36 $
%

\subsection*{NAME}

\noindent VSECSEL ( Virtual Security Selection Protocol )

\input cryptDist

\subsection*{SPECIFICATION}

VSECSEL sits above several lower protocols, selecting them with
security enhancements appropriate to the source and destination
addresses of a session.  In a protocol graph, VSECSEL should sit over a
a key manager instance that can associate an address component with
a security service.


\subsection*{SYNOPSIS}

VSECSEL is only active during connection establishment.  When
openenabled, VSECSEL openenables one of its lower protocols on behalf of
the upper protocol.  When opened, VSECSEL opens a lower protocol, making
its choice based on interactions with the key manager.  The
successfully opened lower session is then returned.

The default service type is the checksum service.

\subsection*{REALM}

Since VSECSEL is active only during open, it should be considered in the
same realm as the protocols below it (which should all be in
the same realm.)

\subsection*{PARTICIPANTS}

On an xOpen, VSECSEL passes the first address component of the remote
participant to the key manager to select the lower security service.
VSECSEL passes its participants to that lower protocol unmodified.  If
that service was an encryption service, then the xOpen of that service
completes, VSECSEL will set the key for both the local and remote
participants to be the same key.

On an xOpenEnable, VSECSEL passes the first address component of the local
participant to the key manager to select the lower security service.
VSECSEL passes its participants to that lower protocol unmodified.

On an xOpenDone, VSECSEL checks that the first address component of
the local participant is compatible with the lower protocol.  For
a lower protocol that is an encryption service, VSECSEL will set
the keys for the local and remote participants to be the same key.
VSECSEL then passes the unmodified participants list to the upper
protocol in xOpenDone.

\subsection*{CONTROL OPERATIONS}

VSECSEL determines the type of control op and forwards it to the
appropriate lower protocol.  For example, CRYPT control ops will be
forwarded to the encryption service.  Other control ops will be
forwarded to all lower services, but the value returned will be from
the first lower protocol.

\subsection*{CONFIGURATION}

The following entry shows how to configure vsecsel over UDP.  The
order of the protocols is crucial and must be correlated with the
key manager file.

\begin{verbatim}
name=vsecsel dir=vsecsel_km files=vsecsel_km protocols=udp,md5,crypt,km/vsec;
\end{verbatim}

A key manager file example is shown below.  It associates UDP port
2001 (decimal) with lower protocol 1, the checksum service; UDP port 2002
with lower protocol 0, the no-security service; UDP port 2003 with
lower protocol 2, the encryption service; UDP port 2004 with the
checksum service.  The encryption service must be the third lower
protocol, because the control operations to set the keys for the
local and remote participants depend on it.  Later versions of the
key manager may allow this restriction to be removed.

\begin{verbatim}
4 4 32
000007d1	00000001
000007d2	00000000
000007d3	00000002
000007d4	00000001
\end{verbatim}

\subsection*{AUTHOR}

\noindent Hilarie Orman
