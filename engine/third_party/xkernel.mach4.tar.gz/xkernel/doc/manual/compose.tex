%
% $Revision: 1.10 $
% $Date: 1992/02/11 23:17:18 $
%

\section{ Composing a kernel: graph.comp }
\label{compose}

A new instance of the $x$-kernel is configured by specifying the
collection of protocols that are to be included in the kernel in a
file called graph.comp (example graph.comp files are in
xkernel/build/Template.)  

The graph.comp file is divided into three sections which describe device
drivers, protocols and miscellaneous configuration parameters. 
The sections are separated by lines beginning with
{\tt @;} and each section may be empty. 

The first two sections, device drivers and protocols, describe the
protocol graph to be configured into your kernel.  Device drivers and
protocols are described by the same types of entries, as in the example:

\font\tt=PS-Courier at 9pt
\begin{verbatim}
name=yap files=yap,yap_clnt,yap_srvr dir=yap, protocols=ip,eth trace=TR_MAJOR_EVENTS;
\end{verbatim}
\font\tt=cmtt10 at 11pt

The first field gives the protocol's name.  The rest of the fields are
optional and may occur in any order.

The {\tt dir} and {\tt files} fields describe the names and
locations of the source files used to build the protocol.  Files are
specified without extensions.  An empty {\tt dir} field will cause
a search for the source files in the current build directory, a
subdirectory in the current build directory with the protocol's name,
and finally in the system protocol library.  An empty {\tt files}
entry defaults to a single {\tt .c} file with the protocols name.
An entry for a protocol from the library should contain neither a
{\tt dir} nor a {\tt files} entry.

The {\tt protocols} field indicates the protocols below the current
protocol in the graph.  When this field contains multiple protocols, order is
significant (the lower protocols will be loaded into the upper
protocol's down vector in the order in which they are listed.)
A protocol which expects multiple protocols below it will describe the
expected semantics of the lower protocols in its manual page in
appendix \ref{protman}.

The {\tt trace} field defines the debugging level used in trace
statements depending on the protocol variable {\tt traceyapp}.  See
section \ref{kdebug} for the symbolic names associated with trace levels.

Multiple instantiations of protocols are supported by using a ``/''
character after the protocol name to add an extension.  In the following
example, two instantiations of ``yap'' are indicated, one over ``ip'' and
one over ``eth,'' and both are used by the ``prt'' protocol:

\font\tt=PS-Courier at 9pt
\begin{verbatim}
name=yaap/ip files=yap,yap_clnt,yap_srvr dir=yap,protocols=ip trace=TR_MAJOR_EVENTS;
name=yaap/eth protocols=eth;
name=prt files=prt dir=prt, protocols=yaap/ip,yaap/eth trace=TR_ERRORS;
\end{verbatim}
\font\tt=cmtt10 at 11pt

Only the first of multiple instantiations should have {\tt dir},
{\tt files}, or {\tt trace} fields.

The third section contains the names of files with protocol number
tables that are to be loaded during initialization.  It also
contains the names of subsystems and their configuration parameters.
Currently trace variables are the only parameters that can be set
here.  The following illustrates a typical use of the third section:

\font\tt=PS-Courier at 9pt
\begin{verbatim}
@;
#
# You can also specify auxiliary protocol tables to be read in at boot
# time.  DEFAULT loads the system protocol table file as listed in etc/site.h
#
prottbl=DEFAULT;
prottbl=./prottbl;

#subsystem tracing for messages and protocol operations
name=msg	trace=TR_GROSS_EVENTS;
name=protocol	trace=TR_MAJOR_EVENTS;
\end{verbatim}
\font\tt=cmtt10 at 11pt

The graph.comp file is read by an $x$-kernel utility program called
{\em compose} which generates startup code to build the protocol
graph and set up the described configuration.  The protocol graph is
built bottom-up; when a protocol's initialization function is called,
the lower level-protocols have already been initialized.


